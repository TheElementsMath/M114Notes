% Options for packages loaded elsewhere
\PassOptionsToPackage{unicode}{hyperref}
\PassOptionsToPackage{hyphens}{url}
%
\documentclass[
]{book}
\usepackage{amsmath,amssymb}
\usepackage{iftex}
\ifPDFTeX
  \usepackage[T1]{fontenc}
  \usepackage[utf8]{inputenc}
  \usepackage{textcomp} % provide euro and other symbols
\else % if luatex or xetex
  \usepackage{unicode-math} % this also loads fontspec
  \defaultfontfeatures{Scale=MatchLowercase}
  \defaultfontfeatures[\rmfamily]{Ligatures=TeX,Scale=1}
\fi
\usepackage{lmodern}
\ifPDFTeX\else
  % xetex/luatex font selection
\fi
% Use upquote if available, for straight quotes in verbatim environments
\IfFileExists{upquote.sty}{\usepackage{upquote}}{}
\IfFileExists{microtype.sty}{% use microtype if available
  \usepackage[]{microtype}
  \UseMicrotypeSet[protrusion]{basicmath} % disable protrusion for tt fonts
}{}
\makeatletter
\@ifundefined{KOMAClassName}{% if non-KOMA class
  \IfFileExists{parskip.sty}{%
    \usepackage{parskip}
  }{% else
    \setlength{\parindent}{0pt}
    \setlength{\parskip}{6pt plus 2pt minus 1pt}}
}{% if KOMA class
  \KOMAoptions{parskip=half}}
\makeatother
\usepackage{xcolor}
\usepackage{longtable,booktabs,array}
\usepackage{calc} % for calculating minipage widths
% Correct order of tables after \paragraph or \subparagraph
\usepackage{etoolbox}
\makeatletter
\patchcmd\longtable{\par}{\if@noskipsec\mbox{}\fi\par}{}{}
\makeatother
% Allow footnotes in longtable head/foot
\IfFileExists{footnotehyper.sty}{\usepackage{footnotehyper}}{\usepackage{footnote}}
\makesavenoteenv{longtable}
\usepackage{graphicx}
\makeatletter
\def\maxwidth{\ifdim\Gin@nat@width>\linewidth\linewidth\else\Gin@nat@width\fi}
\def\maxheight{\ifdim\Gin@nat@height>\textheight\textheight\else\Gin@nat@height\fi}
\makeatother
% Scale images if necessary, so that they will not overflow the page
% margins by default, and it is still possible to overwrite the defaults
% using explicit options in \includegraphics[width, height, ...]{}
\setkeys{Gin}{width=\maxwidth,height=\maxheight,keepaspectratio}
% Set default figure placement to htbp
\makeatletter
\def\fps@figure{htbp}
\makeatother
\setlength{\emergencystretch}{3em} % prevent overfull lines
\providecommand{\tightlist}{%
  \setlength{\itemsep}{0pt}\setlength{\parskip}{0pt}}
\setcounter{secnumdepth}{5}
\usepackage{booktabs}
\ifLuaTeX
  \usepackage{selnolig}  % disable illegal ligatures
\fi
\usepackage[]{natbib}
\bibliographystyle{plainnat}
\usepackage{bookmark}
\IfFileExists{xurl.sty}{\usepackage{xurl}}{} % add URL line breaks if available
\urlstyle{same}
\hypersetup{
  pdftitle={Business Mathematics Course Notes},
  pdfauthor={Joe Hobart},
  hidelinks,
  pdfcreator={LaTeX via pandoc}}

\title{Business Mathematics Course Notes}
\author{Joe Hobart}
\date{}

\begin{document}
\maketitle

{
\setcounter{tocdepth}{1}
\tableofcontents
}
\chapter*{Preamble}\label{preamble}
\addcontentsline{toc}{chapter}{Preamble}

This textbook has been prepared as a companion resource for students enrolled in Business Mathematics at Okanagan College. It is designed to support, not replace, the lecture material delivered in class. While every effort has been made to ensure that the content presented here is accurate and helpful, students are reminded that this resource is not a comprehensive or final source of information. Instead, it should be used alongside classroom instruction, assigned readings, and, most importantly, the official textbook: Business Mathematics by Hobart.

The material in this book covers a broad range of fundamental topics essential to understanding and applying mathematics in a business context. Beginning with a review of basic algebra, the book develops the mathematical skills needed to approach problems in supply and demand economics, cost-volume-profit (CVP) analysis, and merchandising. As students progress, they will explore the time value of money, including calculations involving both simple and compound interest. From there, the book moves into more advanced financial topics such as annuities and mortgages, offering step-by-step procedures and practical examples to illustrate how these concepts apply in real-world business settings.

Each section is written with the aim of being accessible and clear. Wherever possible, examples are contextualized within realistic business scenarios to help students understand not just how to perform calculations, but why they matter in decision-making processes. Practice problems are provided throughout to reinforce learning and encourage independent problem-solving.

Despite the care taken in preparing this resource, students are cautioned that this document may contain errors or omissions. It is not a substitute for professional instruction, nor should it be considered an authoritative source on its own. The official course textbook remains the primary source for course content and should be consulted regularly.

This resource is offered in the spirit of academic support. It is our hope that it will help students build confidence, develop analytical skills, and better understand the mathematical foundations of business decision-making.

\part{Essentials}\label{part-essentials}

\chapter{Preliminaries}\label{preliminaries}

\section*{Lecture 1: The Basics}\label{lecture-1-the-basics}
\addcontentsline{toc}{section}{Lecture 1: The Basics}

\subsection*{Learning Outcomes:}\label{learning-outcomes}
\addcontentsline{toc}{subsection}{Learning Outcomes:}

\begin{enumerate}
\def\labelenumi{\arabic{enumi}.}
\tightlist
\item
  Understand BEDMAS.
\item
  Expand and factor expressions.
\item
  Solve systems of equations by substitution and elimination.
\end{enumerate}

\subsection*{Lecture Notes:}\label{lecture-notes}
\addcontentsline{toc}{subsection}{Lecture Notes:}

Lecture material for this class come from Sections 1.1 -- 1.5 and can be found below. This material is considered review material and so it is not covered in depth.

\begin{enumerate}
\def\labelenumi{\arabic{enumi}.}
\item
  \textbf{Document: \href{https://www.wikihow.com/Set-Decimal-Places-on-a-TI-BA-II-Plus-Calculator}{Setting up my Calculator}}\\
  Your TIBA II Plus financial calculator does not do BEDMAS the way you think it should. You will need to change the settings.
\item
  \textbf{Video: \href{https://youtu.be/ezGExislcbU}{Expanding and Factoring}}\\
  Use BEDMAS to help with order of operations.

  \begin{enumerate}
  \def\labelenumii{\arabic{enumii}.}
  \tightlist
  \item
    Brackets
  \item
    Exponents
  \item
    Multiplication and Division in order from left to right
  \item
    Addition and Subtraction in order from left to right
  \end{enumerate}
\item
  \textbf{Video: \href{https://youtu.be/Yf2jnQHyHdQ}{Fractions}}\\
  Add, subtract, multiply and divide fractions.

  \begin{enumerate}
  \def\labelenumii{\arabic{enumii}.}
  \tightlist
  \item
    \(\frac{a}{b} + \frac{c}{b} = \frac{a+c}{b}\)
  \item
    \(\frac{a}{b} + \frac{c}{d} = \frac{ad}{bd} + \frac{cb}{db} =  \frac{ad + cb}{bd}\)
  \item
    \(\frac{a}{b} \times \frac{c}{d} = \frac{ac}{bd}\)
  \item
    \(\frac{a}{b} \div \frac{c}{d} = \frac{a}{b} \times \frac{d}{c} = \frac{ad}{bc}\)
  \end{enumerate}
\item
  \textbf{Video: \href{https://youtu.be/i8xFdGZ5zOU}{Equations and Expressions I}}\\
  \textbf{Video: \href{https://youtu.be/Vb_2Aqq08ok}{Equations and Expressions II}}\\
  Know how to manipulate equations and expressions.

  \begin{enumerate}
  \def\labelenumii{\arabic{enumii}.}
  \tightlist
  \item
    Solve linear equations for a given variable.
  \item
    Evaluate expressions.
  \item
    Factor and expand a given algebraic expression.
  \end{enumerate}
\item
  \textbf{Video: \href{https://youtu.be/ooUss4RfLHI}{Systems of Equations I}}\\
  \textbf{Video: \href{https://youtu.be/rDDTETJjmyY}{Systems of Equations II}}\\
  You can use the methods of substitution and elimination to solve systems of equations.

  \begin{enumerate}
  \def\labelenumii{\arabic{enumii}.}
  \tightlist
  \item
    Use substitution when it is easy to isolate one of the variables in one of the equations and substitute it into the other equation.
  \item
    Use elimination if the coefficients in front of one of the variables is the same size in both equations.
  \item
    Either method will work for most problems and you are free to use whichever works best for you.
  \end{enumerate}
\end{enumerate}

\subsection*{Lecture Problems:}\label{lecture-problems}
\addcontentsline{toc}{subsection}{Lecture Problems:}

\begin{enumerate}
\def\labelenumi{\arabic{enumi}.}
\tightlist
\item
  Simplify: \[15 + \left[ \frac{3\left(8 + (2 - 10)^2 /4 \right)}{12} \right]^2.\]

  \href{https://youtu.be/3WmpvL441rA}{Click here for the solution}
\item
  Simplify and collect like terms: \[\frac{8x}{0.5} + \frac{5.5x}{11} + 0.5\left(4.6x - 17 \right).\]

  \href{https://youtu.be/BHPP---sVZw}{Click here for the solution}
\item
  Solve the following: \[\frac{x}{1.1^2} + 2x(1.1)^3 = \$1000.\]

  \href{https://youtu.be/PZqTWdq1wiI}{Click here for the solution}
\item
  Solve the system of equations: \begin{align*}     3x + 4y &= 55\\  10x + 5y &= 100.\end{align*}

  \href{https://youtu.be/LvRuXkQ5Mao}{Click here for the solution}
\item
  Solve the system of equations: \begin{align*}     10x - 6y &= -10 \\ 4x + 6y &= 38.\end{align*}

  \href{https://youtu.be/YQjb2Z6y51E}{Click here for the solution}
\item
  Solve the system of equations: \begin{align*}     4x - 5y &= 18 \\ 4x + 3y &= 2.\end{align*}

  \href{https://youtu.be/c_YPN8Xt0Fc}{Click here for the solution}
\end{enumerate}

\subsection*{Additional Problems:}\label{additional-problems}
\addcontentsline{toc}{subsection}{Additional Problems:}

Additional problems that are typically done in class (with video solutions) can be found here:

\begin{itemize}
\item
  \href{https://theelementsmath.github.io/M114/preliminaries.html\#order-of-operations}{BEDMAS}
\item
  \href{https://theelementsmath.github.io/M114/preliminaries.html\#rational-numbers}{Rational Numbers}
\item
  \href{https://theelementsmath.github.io/M114/preliminaries.html\#expanding-and-factoring}{Expanding and Factoring}
\item
  \href{https://theelementsmath.github.io/M114/preliminaries.html\#solving-systems-of-equations}{Solving Systems of Equations}
\end{itemize}

\section*{Lecture 2: Ratios, Rates, Proportions and Percentages}\label{lecture-2-ratios-rates-proportions-and-percentages}
\addcontentsline{toc}{section}{Lecture 2: Ratios, Rates, Proportions and Percentages}

\subsection*{Learning Outcomes:}\label{learning-outcomes-1}
\addcontentsline{toc}{subsection}{Learning Outcomes:}

\begin{enumerate}
\def\labelenumi{\arabic{enumi}.}
\tightlist
\item
  Interpret and apply rates.
\item
  Understand and use percentages.
\item
  Work with fractions and decimals.
\item
  Convert from rates/ percentages to ratios to fractions/ decimals.
\item
  Solve problems Involving ratios, rates, and percentages.
\end{enumerate}

\subsection*{Review Problems From Last Lecture:}\label{review-problems-from-last-lecture}
\addcontentsline{toc}{subsection}{Review Problems From Last Lecture:}

\begin{enumerate}
\def\labelenumi{\arabic{enumi}.}
\tightlist
\item
  Simplify the expression: \(\;\;\;\;\; 900(1 + 0.05(15/12))\).{]}

  \href{https://youtu.be/zgLRtB3JUpE}{Click here for the solution}
\item
  Simplify the expression: \(\;\;\;\;\; 300 \left[\dfrac{(1 + 0.08/2)^{30} - 1}{0.08/2} \right]\).

  \href{https://youtu.be/Emv6osLWEvE}{Click here for the solution}
\item
  Solve the following equation for \(x\): \(\;\;\;\;\; 15 x - \frac{5}{2} = \frac{15}{2}x + 10\).

  \href{https://youtu.be/3WmpvL441rA}{Click here for the solution}
\item
  Solve the following system of equations: \begin{align*} x + y &= 13\\ 2x - y &=8 \end{align*}

  \href{https://youtu.be/57H_RLVgpc8}{Click here for the solution}
\end{enumerate}

\subsection*{Lecture Notes:}\label{lecture-notes-1}
\addcontentsline{toc}{subsection}{Lecture Notes:}

Lecture material for this class come from Sections 1.6 and can be found below. This material is considered review material and so it is not covered in depth.

\begin{enumerate}
\def\labelenumi{\arabic{enumi}.}
\tightlist
\item
  \textbf{Video: \href{https://www.youtube.com/watch?v=GU3i-S4sxdw}{Percentages}}\\
  \textbf{Video: \href{https://www.youtube.com/watch?v=Nj1cYFE7cBA}{Ratios}}\\
  \textbf{Video: \href{https://www.youtube.com/watch?v=j4TboDA7cNI}{Proportions}}\\
  Ratios are used to compare multiple quantities. They can be expressed: (a) using a colon, (b) as a fraction, (c) as a decimal, and (d) as a percentage.

  \begin{itemize}
  \tightlist
  \item
    \textbf{Ratios:} A comparison between two quantities using division.\\
    Example: A ratio of 3 to 2 can be written as \(3:2\), \(\frac{3}{2}\), or ``3 to 2''.
  \item
    \textbf{Rates:} A specific type of ratio comparing quantities with different units.\\
    Example: A car travels 60 miles in 2 hours, so the rate is \(\frac{60 \text{ miles}}{2 \text{ hours}} = 30 \text{ miles/hour}\).
  \item
    \textbf{Fractions:} Express part of a whole as a ratio of two numbers (numerator and denominator).\\
    Example: \(\frac{1}{4}\) means 1 part out of 4 equal parts.
  \item
    \textbf{Percentages:} A fraction with a denominator of 100. Expresses how many parts out of 100.\\
    Example: \(25\% = \frac{25}{100} = 0.25\)
  \item
    \textbf{Decimals:} A way of expressing fractions and percentages using powers of 10.\\
    Example: \(0.5 = \frac{1}{2} = 50\%\)
  \end{itemize}
\item
  \textbf{Video: \href{https://www.youtube.com/watch?v=O9YKlxowu68}{Taxes}}\\
  \textbf{Video: \href{https://www.youtube.com/watch?v=ChgppEgOYc0}{Exchange Rates}}\\
  Ratios, rates and proportions have many different applications. They allow us to compare quantities, scale values, and interpret data in meaningful ways.

  \begin{itemize}
  \tightlist
  \item
    \textbf{Ratios in Simplest Integer Form:} In many contexts, we prefer to express ratios using whole numbers for simplicity and clarity.

    \begin{itemize}
    \tightlist
    \item
      Example: A recipe that uses 2 parts sugar to 3 parts flour can be written as the ratio \(2:3\).
    \end{itemize}
  \item
    \textbf{Ratios with a term of 1:} It is often helpful to express a ratio with one of the terms as 1 to make comparisons easier.

    \begin{itemize}
    \tightlist
    \item
      Example: If the ratio of students to teachers is \(24:1\), it means there are 24 students per teacher.
    \end{itemize}
  \item
    \textbf{Multiple Ratios or Rates in One Situation:} Some problems involve more than two quantities or rates that must be compared simultaneously.

    \begin{itemize}
    \tightlist
    \item
      Example: A mixture of paint may require a ratio of red : blue : white = \(3:2:5\).\\
    \item
      In finance, one might compare exchange rates between three currencies: USD to EUR, EUR to GBP, and USD to GBP.
    \end{itemize}
  \item
    \textbf{Real-World Applications:} Currency, Tax, and Conversions: Ratios and rates are commonly used in practical calculations such as:

    \begin{itemize}
    \tightlist
    \item
      Exchange Rates: Converting currencies (e.g., \(1 \text{ USD} = 0.92 \text{ EUR}\)).
    \item
      Taxes and Discounts: Applying percentage increases or decreases to prices (e.g., adding 15\% tax, taking off 20\% discount).
    \item
      Unit Conversions: Converting between units using rates (e.g., \(1 \text{ inch} = 2.54 \text{ cm}\)).
    \end{itemize}
  \end{itemize}
\end{enumerate}

\subsection*{Lecture Problems:}\label{lecture-problems-1}
\addcontentsline{toc}{subsection}{Lecture Problems:}

\begin{enumerate}
\def\labelenumi{\arabic{enumi}.}
\tightlist
\item
  Percentages

  \begin{enumerate}
  \def\labelenumii{\alph{enumii}.}
  \tightlist
  \item
    Calculate 175\% of \$500.
  \item
    65\% of what amount is \$85?
  \item
    The tax rate is 8.25\%. How much tax is required for a purchase of \$450?

    \href{https://youtu.be/C5Ik5tzzrtE}{Click here for the solution}
  \end{enumerate}
\item
  The accounting, mathematics and economics departments have 10, 14 and 8 employees respectively. We have a \$10,000 marketing budget that needs to be divided among those departments based on the ratio of the number of employees. How much should each department receive?

  \href{https://youtu.be/2CGmE2oImEk}{Click here for the solution}
\item
  The accounting, mathematics and economics departments have 10, 14 and 8 employees respectively. The departments teach 500, 1000 and 350 students respectively. Which department is generating the most student revenue compared to their department size?

  \href{https://youtu.be/FLtoyc6iuGY}{Click here for the solution}
\item
  Jim and Ben have a partnership with a 40-60 split. Jim is going to invest an additional \$20,000 into the partnership. How much should Ben invest?

  \href{https://youtu.be/v_de0pgxKco}{Click here for the solution}
\end{enumerate}

\subsection*{Additional Problems:}\label{additional-problems-1}
\addcontentsline{toc}{subsection}{Additional Problems:}

Additional problems that are typically done in class (with video solutions) can be found here:

\begin{itemize}
\tightlist
\item
  \href{https://theelementsmath.github.io/M114/preliminaries.html\#ratios-proportions-and-percentages}{Ratios, Proportions and Percentages}
\end{itemize}

\part{Economics}\label{part-economics}

\chapter{Introduction to Microeconomics}\label{introduction-to-microeconomics}

\section*{Lecture 3: Supply and Demand Economics}\label{lecture-3-supply-and-demand-economics}
\addcontentsline{toc}{section}{Lecture 3: Supply and Demand Economics}

\subsection*{Learning Outcomes:}\label{learning-outcomes-2}
\addcontentsline{toc}{subsection}{Learning Outcomes:}

\begin{enumerate}
\def\labelenumi{\arabic{enumi}.}
\tightlist
\item
  Find the equilibrium price and quantity.
\item
  Identify situations of shortage and excess.
\item
  Solve problems involving shocks to the economy.
\end{enumerate}

\subsection*{Review Problems From Last Lecture:}\label{review-problems-from-last-lecture-1}
\addcontentsline{toc}{subsection}{Review Problems From Last Lecture:}

\begin{enumerate}
\def\labelenumi{\arabic{enumi}.}
\tightlist
\item
  You purchase a jacket for \$245 after tax. The original price of the jacket was \$215. What is the tax rate?

  \href{https://youtu.be/YnkjwUr_YFQ}{Click here for the solution}
\item
  There are 300 textbooks sold for every 357 students. If enrollment increases by 15\%, what is the new enrolment and how many textbooks are needed?

  \href{https://youtu.be/clGY-9mrAFU}{Click here for the solution}
\item
  The OCSU wants to raise \$1.25 million over the next 5 years to fund a student union building. With current enrolments at 3000 students with an expected increase of 5\% over the previous year, how much should OCSU charge each student in order to raise this money?

  \href{https://youtu.be/Aum_ZyukTHM}{Click here for the solution}
\item
  Abby and Max are starting a partnership where Abby will originally invest \$35,000 and Max will invest \$75,000. Abby wants to even out the ratio, so she adds another \$20,000 to the investment. If Max wanted to maintain the original ratio, how much would he have to add to the investment?

  \href{https://youtu.be/vrGvTSZJ8uI}{Click here for the solution}
\item
  Lacey bought 25.5 hectares of land for \$63,000. If she buys another 80 hectares of land at the same price per hectare as her first purchase, how much must she pay for her second purchase?

  \href{https://youtu.be/PSSyIpvduZs}{Click here for the solution}
\end{enumerate}

\subsection*{Lecture Notes:}\label{lecture-notes-2}
\addcontentsline{toc}{subsection}{Lecture Notes:}

Lecture material for this class come from Section 2.2 and can be found below. This material is considered review material and so it is not covered in depth.

\begin{enumerate}
\def\labelenumi{\arabic{enumi}.}
\tightlist
\item
  \textbf{Video: \href{ttps://www.bing.com/videos/riverview/relatedvideo?q=youtube+\%2b+assumptions+in+microeconomics&mid=9CED92F92F430E9D1F069CED92F92F430E9D1F06&FORM=VIRE}{Supply and Demand Economics}}\\
  There are several assumptions that are typically made in a perfectly competitive microeconomic market.

  \begin{itemize}
  \tightlist
  \item
    \textbf{Rational Behavior:} Individuals are rational and aim to maximize their utility or profit.
  \item
    \textbf{Profit Maximization:} Firms choose output levels to maximize profit, typically where marginal cost equals marginal revenue.

    \begin{itemize}
    \tightlist
    \item
      Perfect Information: All agents have full and accurate information for decision-making.
    \item
      Ceteris Paribus: All other factors are held constant when analyzing a change in one variable.
    \item
      Perfect Competition: Many buyers and sellers, identical products, no market power, and free market entry/exit.\\
    \end{itemize}
  \item
    \textbf{No Externalities:} All costs and benefits are internal to market participants; no spillover effects.
  \end{itemize}
\item
  \textbf{Video: \href{https://www.bing.com/videos/riverview/relatedvideo?&q=youtube+\%2b+the+basics+of+the+supply+curve&&mid=0B4E6D4CB20ADD15E9600B4E6D4CB20ADD15E960&&FORM=VRDGAR}{The Basics of the Supply Curve}}

  \begin{itemize}
  \tightlist
  \item
    \textbf{Upward Sloping:} The supply curve generally slopes upward, indicating that higher prices incentivize producers to supply more.
  \item
    \textbf{Movement Along the Curve:} Caused by a change in the price of the good itself. This reflects a change in quantity supplied.
  \item
    \textbf{Short Run vs.~Long Run:} In the short run, some inputs are fixed; in the long run, all inputs are variable, leading to a potentially more elastic supply.
  \end{itemize}
\item
  \textbf{Video: \href{https://www.youtube.com/watch?v=kUPm2tMCbGE}{The Basics of the Demand Curve}}

  \begin{itemize}
  \tightlist
  \item
    \textbf{Downward Sloping:} The demand curve typically slopes downward, indicating that as the price of a good decreases, the quantity demanded increases.
  \item
    \textbf{Movement Along the Curve:} Caused by a change in the price of the good itself. This reflects a change in quantity demanded, not demand.
  \end{itemize}
\item
  \textbf{Video: \href{https://www.youtube.com/watch?v=7eZcPs9z9OA}{Excesses and Surplusses}}

  \begin{itemize}
  \tightlist
  \item
    \textbf{Shortage (Excess Demand):} Occurs when the quantity demanded exceeds the quantity supplied at a given price.\\
  \item
    \textbf{Caused by Price Below Equilibrium:} If the market price is set below the equilibrium price, demand increases and supply decreases, creating a shortage.\\
  \item
    \textbf{Upward Pressure on Price:} In a shortage, consumers compete for limited goods, pushing prices upward toward equilibrium.
  \item
    \textbf{Surplus (Excess Supply):} Occurs when the quantity supplied exceeds the quantity demanded at a given price.
  \item
    \textbf{Caused by Price Above Equilibrium:} If the market price is set above the equilibrium price, supply increases and demand decreases, creating a surplus.
  \item
    \textbf{Downward Pressure on Price:} In a surplus, sellers lower prices to clear excess inventory, moving the market toward equilibrium.
  \item
    \textbf{Market Forces Restore Equilibrium:} In a competitive market, shortages and surpluses are typically temporary, as price adjustments lead the market back to equilibrium.
  \end{itemize}
\item
  \textbf{Video: \href{https://www.youtube.com/watch?v=6dwSIO_Slhc}{Shocks to Equilibrium}}

  \begin{itemize}
  \tightlist
  \item
    \textbf{Shift vs.~Movement:} A shift refers to a change in the entire supply or demand curve, while a movement refers to a change along the curve due to a price change.
  \item
    \textbf{Demand Curve Shifts:} Occur when a non-price determinant of demand changes. These include:

    \begin{itemize}
    \tightlist
    \item
      Income: Increases in income shift the demand for normal goods right and for inferior goods left.
    \item
      Prices of Related Goods:

      \begin{itemize}
      \tightlist
      \item
        Substitutes: An increase in the price of one increases demand for the other.
      \item
        Complements: An increase in the price of one decreases demand for the other.
      \end{itemize}
    \item
      Tastes and Preferences: Favorable changes shift demand right.
    \item
      Expectations: Expectations of future price increases can raise current demand.
    \item
      Number of Buyers: More buyers increase market demand.
    \end{itemize}
  \item
    \textbf{Supply Curve Shifts:} Occur when a non-price determinant of supply changes. These include:

    \begin{itemize}
    \tightlist
    \item
      Input Prices: Lower input costs shift supply right; higher costs shift it left.
    \item
      Technology: Improvements shift supply right by increasing efficiency.
    \item
      Expectations: If future prices are expected to rise, current supply may decrease.
    \item
      Number of Sellers: More sellers shift supply right.
    \item
      Government Policies: Taxes, subsidies, and regulations can shift supply.
    \end{itemize}
  \item
    \textbf{Effect on Equilibrium:}

    \begin{itemize}
    \tightlist
    \item
      Demand Increase: Raises equilibrium price and quantity.
    \item
      Demand Decrease: Lowers equilibrium price and quantity.
    \item
      Supply Increase: Lowers equilibrium price and raises quantity.
    \item
      Supply Decrease: Raises equilibrium price and lowers quantity.
    \end{itemize}
  \end{itemize}
\end{enumerate}

\subsection*{Lecture Problems:}\label{lecture-problems-2}
\addcontentsline{toc}{subsection}{Lecture Problems:}

\begin{enumerate}
\def\labelenumi{\arabic{enumi}.}
\tightlist
\item
  Suppose the market demand and supply for a certain good are given by the following equations:
  \begin{align*}
  Q_D &= 100 - 2P \quad (Demand) \\
  Q_S &= 20 + 3P \quad (Supply)
  \end{align*}
  Where:

  \begin{itemize}
  \tightlist
  \item
    \(Q_D\) is the quantity demanded,
  \item
    \(Q_S\) is the quantity supplied,
  \item
    \(P\) is the price of the good.
    Find the equilibrium price and equilibrium quantity in this market.

    \href{https://youtu.be/Abb7bN8yySg}{Click here for the solution}
  \end{itemize}
\item
  Suppose the market demand and supply for a product are given by the following equations:
  \begin{align*}
  Q_D &= 80 - 2P \quad (Demand) \\
  Q_S &= 20 + 3P \quad (Supply)
  \end{align*}
  Assume the current market price is \(P = 10\).

  \begin{itemize}
  \tightlist
  \item
    Calculate the quantity demanded and quantity supplied at this price.
  \item
    Is the market experiencing a shortage, a surplus, or is it in equilibrium?
  \item
    Based on your answer, what is likely to happen to the price in the next period?

    \href{\%7Bhttps://youtu.be/SKh2EvI4YLc\%7D}{Click here for the solution}
  \end{itemize}
\item
  A foodborne illness outbreak reduces consumer confidence in a particular brand of lettuce.

  \begin{itemize}
  \tightlist
  \item
    What happens to the \textcolor{red}{demand curve} for that lettuce brand?
  \item
    What is the likely market outcome for price and quantity?

    \href{https://youtu.be/ChjAjdemwR4}{Click here for the solution}
  \end{itemize}
\item
  Due to a surge in health consciousness, consumers increasingly prefer plant-based protein, leading to an increase in the demand for tofu. At the same time, technological improvements reduce production costs, increasing the supply of tofu.

  \begin{itemize}
  \tightlist
  \item
    Illustrate the shifts in both the demand and supply curves on a graph.
  \item
    What is the expected effect on the \textcolor{blue}{equilibrium quantity}?
  \item
    Can we determine the effect on the \textcolor{blue}{equilibrium price}? Why or why not?

    \href{https://youtu.be/RlajAIMXlyA}{Click here for the solution}
  \end{itemize}
\end{enumerate}

\subsection*{Additional Problems:}\label{additional-problems-2}
\addcontentsline{toc}{subsection}{Additional Problems:}

Additional problems that are typically done in class (with video solutions) can be found here:

\begin{itemize}
\tightlist
\item
  \href{https://theelementsmath.github.io/M114/introduction-to-microeconomics.html\#supply-and-demand-economics}{Supply and Demand Economics}
\end{itemize}

\part{Economics}\label{part-economics-1}

\chapter{Business Economics}\label{business-economics}

\section*{Lecture 4: Merchandizing}\label{lecture-4-merchandizing}
\addcontentsline{toc}{section}{Lecture 4: Merchandizing}

\subsection*{Learning Outcomes:}\label{learning-outcomes-3}
\addcontentsline{toc}{subsection}{Learning Outcomes:}

\begin{enumerate}
\def\labelenumi{\arabic{enumi}.}
\tightlist
\item
  Find net price, list price and trade discount given the other variables.
\item
  Find a single equivalent trade discount.
\item
  Find the markup, selling price, and cost given the other variables.
\item
  Find the markdown, rate of markdown and selling price given the other variables.
\item
  Find the rate of gross profit margin (rate of markup on cost) and the rate of markup (on price).
\end{enumerate}

\subsection*{Review Problems From Last Lecture:}\label{review-problems-from-last-lecture-2}
\addcontentsline{toc}{subsection}{Review Problems From Last Lecture:}

\begin{enumerate}
\def\labelenumi{\arabic{enumi}.}
\tightlist
\item
  Coke has recently lowered its price in order to grab more of the market share for cola. What will happen to the equilibrium price and quantity for ``other colas'' as a result of this action by Coke?

  \href{https://youtu.be/CEzvcP93ARo}{Click here for the solution}
\item
  You are a farmer who has recently discovered a way to increase your annual yield of grapes. You end up selling your idea to all of the other farmers in the area. What happens to the equilibrium price and quantity for grapes as a result?

  \href{https://youtu.be/s6vObZi1PuY}{Click here for the solution}
\item
  As a wine maker, the price of grapes has recently decreased (see question \#2). How does this change the equilibrium price and quantity for wine?

  \href{https://youtu.be/ub-yZFjkZO4}{Click here for the solution}
\item
  You run a fancy restaurant and one of your best selling combinations is your fruit and cheese platter (as it goes very well with wine). The price of wine has decreased recently, what effect does this have on your fruit and wine platter sales?

  \href{https://youtu.be/t7KJD_0uoL0}{Click here for the solution}
\item
  The market for electric bicycles (e-bikes) has recently experienced two major changes: (a) Due to rising fuel prices and increased environmental awareness, more consumers are choosing e-bikes as their primary mode of transportation. (b) At the same time, a global shortage of lithium (a key component in e-bike batteries) has increased production costs and limited the number of e-bikes manufacturers can produce. Using a supply and demand diagram, what happens to the equilibrium price and equilibrium quantity of e-bikes?

  \href{https://youtu.be/rXxkmAUtmQI}{Click here for the solution}
\end{enumerate}

\subsection*{Lecture Notes:}\label{lecture-notes-3}
\addcontentsline{toc}{subsection}{Lecture Notes:}

Lecture material for this class come from Section 3.2 and can be found below. This material is considered review material and so it is not covered in depth.

\begin{enumerate}
\def\labelenumi{\arabic{enumi}.}
\tightlist
\item
  \textbf{Video: \href{https://youtu.be/NBrUqBwk5sE}{Trade Discounts}}\\
  The net price is the result of reducing the list price by the trade discount.

  \begin{itemize}
  \tightlist
  \item
    \textbf{List Price (Marked Price):} The original price of a product before any discounts are applied. This is the price set by the manufacturer or seller.
  \item
    \textbf{Trade Discount:} A percentage reduction from the list price offered by the seller to buyers, often based on quantities purchased or buyer relationships. It is not usually recorded in the books.
  \item
    \textbf{Net Price:} The actual amount the buyer pays after the trade discount is applied. It is calculated as:
    \[ \text{Net Price} = \text{List Price} \times (1 - \text{Trade Discount}) \]
  \end{itemize}
\item
  \textbf{Video: \href{https://youtu.be/IXx3Ix2rr8Y}{Markups}}\\
  To ensure the business covers its operating expenses and earns a profit, it adds a markup to the cost.

  \begin{itemize}
  \tightlist
  \item
    \textbf{Cost:} The total expenditure incurred in producing or purchasing a product. This includes materials, labor, and other direct expenses.
  \item
    \textbf{Markup:} The amount added to the cost to determine the selling price. Markup covers both profit and overhead (indirect costs such as rent, utilities, administrative expenses).

    \begin{itemize}
    \tightlist
    \item
      \textbf{Overhead:} Indirect costs of operating the business, not directly tied to a specific product.
    \item
      \textbf{Profit:} The financial gain remaining after all expenses have been covered.

      \begin{itemize}
      \tightlist
      \item
        Markup is given by: \[ \text{Markup} = \text{Overhead} + \text{Profit}.\]
      \end{itemize}
    \end{itemize}
  \item
    \textbf{Selling Price:} The final price at which the product is sold to customers. It includes both the cost and the markup. It can be expressed as:
    \begin{align*}
     \text{Selling Price} &= \text{Cost} + \text{Markup}\\
     \text{Selling Price} &= \text{Cost} + \text{Overhead} + \text{Profit}.
     \end{align*}
  \end{itemize}
\item
  \textbf{Video: \href{https://youtu.be/GfI57rP47R4}{Markdowns}}\\
  The sale price is a reduced form of the selling price, directly influenced by the markdown rate.

  \begin{itemize}
  \tightlist
  \item
    \textbf{Rate of Markdown:} The percentage reduction applied to the selling price to encourage sales, clear inventory, or match competition. It is usually expressed as a percentage:
    \[
     \text{Rate of Markdown} = \frac{\text{Selling Price} - \text{Sale Price}}{\text{Selling Price}} \times 100\%
     \]
  \item
    \textbf{Sale Price:} The final price paid by the customer after the markdown is applied. It can be calculated as:
    \[
     \text{Sale Price} = \text{Selling Price} \times (1 - \text{Markdown Rate})
     \]
  \end{itemize}
\item
  Markup on Cost vs.~Gross Profit Margin.

  \begin{itemize}
  \tightlist
  \item
    \textbf{Rate of Markup on Cost:} This measures the percentage increase from the cost price to the selling price. It tells how much profit is made based on the cost of the item. The formula is:
    \[
     \text{Markup Rate (on Cost)} = \frac{\text{Selling Price} - \text{Cost}}{\text{Cost}} \times 100\%
     \]
  \item
    \textbf{Gross Profit Margin (Rate of Markup on Selling Price):} This measures the profit as a percentage of the selling price. It reflects how much of each dollar of sales is profit before deducting operating expenses. The formula is:
    \[
     \text{Gross Profit Margin} = \frac{\text{Selling Price} - \text{Cost}}{\text{Selling Price}} \times 100\%
     \]
  \item
    Key Difference:

    \begin{itemize}
    \tightlist
    \item
      The markup rate is based on the cost, while the gross profit margin is based on the selling price.
    \item
      For the same product, the markup rate will always be higher than the gross profit margin.
    \end{itemize}
  \end{itemize}
\end{enumerate}

\subsection*{Lecture Problems:}\label{lecture-problems-3}
\addcontentsline{toc}{subsection}{Lecture Problems:}

\begin{enumerate}
\def\labelenumi{\arabic{enumi}.}
\tightlist
\item
  A product has a list price of \$1,000. It is offered with two successive trade discounts of 20\% and 10\%. Calculate the net price after applying both trade discounts.

  \href{https://youtu.be/RYQOeGDUyv0}{Click here for the solution}
\item
  If the cost of an item is \$180 and the seller applies a markup rate of 20\% on cost for overhead and a 15\% markup on selling price for profit, what is the selling price of the item?

  \href{https://youtu.be/8D4lgsp54FI}{Click here for the solution}
\item
  An item originally priced at \$200 is marked down by 25\%. What is the sale price after the markdown?

  \href{https://youtu.be/6aA6vWWjNCM}{Click here for the solution}
\item
  A product is sold for \$180 after a 20\% markdown on its original selling price. If the cost of the product is \$120, calculate:

  \begin{itemize}
  \tightlist
  \item
    The original selling price before the markdown,
  \item
    The gross profit margin based on the sale price,
  \item
    The rate of markup based on the sale price.

    \href{https://youtu.be/W7gbRh4Z1YQ}{Click here for the solution}
  \end{itemize}
\item
  A wholesaler lists a product at \$500 with trade discounts of 15\% and 5\%. The net price after discounts is then marked up by 30\% to determine the selling price. Later, the retailer applies a 10\% markdown on the selling price to promote sales. If the cost to the retailer is equal to the wholesaler's selling price, calculate:

  \begin{itemize}
  \tightlist
  \item
    The net price after trade discounts,
  \item
    The retailer's selling price after markup,
  \item
    The rate of markup and gross profit margin based on the original selling price,
  \item
    The final sale price after markdown,
  \item
    The total markdown, and
  \item
    The rate of markup and gross profit margin based on the final sale price.

    \href{https://youtu.be/03v9Et4Do5w}{Click here for the solution}
  \end{itemize}
\end{enumerate}

\subsection*{Additional Problems:}\label{additional-problems-3}
\addcontentsline{toc}{subsection}{Additional Problems:}

Additional problems that are typically done in class (with video solutions) can be found here:

\begin{itemize}
\tightlist
\item
  \href{https://theelementsmath.github.io/M114/business-economics.html\#merchandizing}{Merchandizing}
\end{itemize}

\section*{Lecture 5: Cost, Volume, Profit Analysis}\label{lecture-5-cost-volume-profit-analysis}
\addcontentsline{toc}{section}{Lecture 5: Cost, Volume, Profit Analysis}

\subsection*{Learning Outcomes:}\label{learning-outcomes-4}
\addcontentsline{toc}{subsection}{Learning Outcomes:}

\begin{enumerate}
\def\labelenumi{\arabic{enumi}.}
\tightlist
\item
  Define and explain the key components of cost-volume-profit (CVP) analysis, including fixed costs, variable costs, contribution margin, and total revenue.
\item
  Calculate the contribution margin per unit and contribution margin ratio.
\item
  Determine the breakeven level of output in units and in sales revenue using both the equation and contribution margin methods.
\item
  Illustrate CVP relationships using breakeven charts and profit-volume graphs.
\item
  Distinguish between the breakeven point and the shutdown point.
\item
  Determine the shutdown point and explain its relevance in short-run production decisions.
\end{enumerate}

\subsection*{Review Problems From Last Lecture:}\label{review-problems-from-last-lecture-3}
\addcontentsline{toc}{subsection}{Review Problems From Last Lecture:}

\begin{enumerate}
\def\labelenumi{\arabic{enumi}.}
\tightlist
\item
  A business purchases pre-hung doors for a list price of \$350 with trade discounts of 30\% and 14\%. The business then marks the doors up 20\% of selling price for overhead and 25\% of cost for profit.

  \begin{itemize}
  \tightlist
  \item
    What is the net price of the doors?
  \item
    What is the selling price of the doors?
  \item
    What is the ROMU (Rate of Markup on Unit Selling Price)?
  \item
    What is the GPM (Gross Profit Margin)?

    \href{https://youtu.be/CMOw7ktpKgk}{Click here for the solution}
  \end{itemize}
\item
  During a sale, the doors are marked down to sell at cost. What is the \textbf{ROMD} (Rate of Markdown on Selling Price)?

  \href{https://youtu.be/rZGY5cLtock}{Click here for the solution}
\item
  During a different sale, the doors are marked down so that the business breaks even. In this case, what is the \textbf{ROMD}?

  \href{https://youtu.be/BIgdVkHfEmc}{Click here for the solution}
\item
  A business has a ROMU of 72\%. The business sells power tools for \$392.

  \begin{itemize}
  \tightlist
  \item
    What is the GPM?
  \item
    What is the cost?
  \item
    What is the markup?

    \href{https://youtu.be/En9picJQfME}{Click here for the solution}
  \end{itemize}
\end{enumerate}

\subsection*{Lecture Notes:}\label{lecture-notes-4}
\addcontentsline{toc}{subsection}{Lecture Notes:}

Lecture material for this class come from Sections 3.3 and can be found below. This material is considered review material and so it is not covered in depth.

\begin{enumerate}
\def\labelenumi{\arabic{enumi}.}
\tightlist
\item
  \textbf{Video: \href{https://youtu.be/c5i7FsLeDk8}{Profit Functions}}

  \begin{itemize}
  \tightlist
  \item
    \textbf{Fixed Costs (FC):} Fixed costs are costs that do not change with output.
  \item
    \textbf{Total Variable Costs (TVC):} Total variable costs are costs that change as output changes.
  \item
    \textbf{Variable Costs Per Unit (VC):} Variable costs per unit are the amount that costs increase when we increase output by 1 unit.
  \item
    \textbf{Total Costs (TC):} The sum of fixed costs and variable costs \[ \text{Total Cost} = FC + TVC\]
  \item
    \textbf{Profit Function:}
    \begin{align*}
     \text{Profit} &= \text{Total Revenue} - \text{Total Cost} \\
     &= (P \cdot Q) - (FC + TVC)\\
     &= P \cdot Q - VC \cdot Q - FC\\
     &= (P-VC) \cdot Q - FC
     \end{align*}
    where \(P\) = price per unit, \(Q\) = quantity sold, \(FC\) = fixed costs, \(VC\) = variable cost per unit, and \(TVC\) = total variable costs.
  \end{itemize}
\item
  \textbf{Video: \href{https://youtu.be/RMwndvVD894}{Contribution Margin/ Rate}}

  \begin{itemize}
  \tightlist
  \item
    \textbf{Contribution Margin:} The amount each unit contributes to covering fixed costs and generating profit.
    \[
     \text{Contribution Margin} = P - VC
     \]
  \item
    \textbf{Contribution Rate (or Ratio):} The proportion of each sales dollar that contributes to covering fixed costs and profit.
    \[
     \text{Contribution Rate} = \frac{P - VC}{P} = \frac{\text{Contribution Margin}}{\text{Selling Price}}
     \]
  \end{itemize}
\item
  \textbf{Video: \href{https://youtu.be/td4Veo3q4r4}{Break-even Point}}

  \begin{itemize}
  \tightlist
  \item
    The level of output at which total revenue equals total costs (i.e., profit is zero):
    \[
     \text{Break-even Quantity} = \frac{FC}{\text{Contribution Margin per Unit}}
     \]
  \end{itemize}
\item
  \textbf{Video: \href{https://youtu.be/td4Veo3q4r4}{Shutdown Point}}

  \begin{itemize}
  \tightlist
  \item
    The shutdown point is the output level where the contribution margin is 0. The level of output where price equals the variable cost per unit (VC). If price falls below VC, the firm should shut down in the short run.
    \[
     \text{Shutdown Point: } CM < 0
     \]
    \[
     \text{Shutdown Point: } P < VC
     \]
  \end{itemize}
\end{enumerate}

\subsection*{Lecture Problems:}\label{lecture-problems-4}
\addcontentsline{toc}{subsection}{Lecture Problems:}

\begin{enumerate}
\def\labelenumi{\arabic{enumi}.}
\tightlist
\item
  A business sells a product for \$80 per unit. The variable cost per unit is \$50 and fixed costs are \$12,000.

  \begin{itemize}
  \tightlist
  \item
    Write the profit function.
  \item
    Calculate profit if the business sells 400 units.
  \item
    How many units must be sold to earn a profit of \$8,000?

    \href{https://youtu.be/ZkAqt5r75JE}{Click here for the solution}
  \end{itemize}
\item
  A company incurs \$18,000 in fixed costs. It sells a product for \$60 per unit and incurs a variable cost of \$35 per unit.

  \begin{itemize}
  \tightlist
  \item
    What is the contribution margin per unit?
  \item
    How many units must be sold to break even?
  \item
    If the company sells 1,000 units, what is its total profit?

    \href{https://youtu.be/IpD_11jJQT8}{Click here for the solution}
  \end{itemize}
\item
  A bakery has fixed costs of \$10,500 per month. Each loaf of bread sells for \$6 and has a variable cost of \$2.50.

  \begin{itemize}
  \tightlist
  \item
    Find the break-even number of loaves
  \item
    What is the total revenue at break-even?
  \item
    What is the profit if the bakery sells 3,500 loaves?

    \href{https://youtu.be/P8KLl_D-kNM}{Click here for the solution}
  \end{itemize}
\item
  A product sells for \$75 and has a variable cost of \$45.

  \begin{itemize}
  \tightlist
  \item
    What is the contribution margin?
  \item
    What is the contribution rate (as a percentage)?
  \item
    How many units are required to cover fixed costs of \$15,000?

    \href{https://youtu.be/ONvoWKJxvLM}{Click here for the solution}
  \end{itemize}
\item
  A company sells gadgets at \$100 each. Variable cost per gadget is \$60. Fixed costs are \$20,000.

  \begin{itemize}
  \tightlist
  \item
    Calculate the break-even quantity.
  \item
    If the firm wants to earn a profit of \$10,000, how many units must it sell?
  \item
    What is the profit if 400 units are sold?

    \href{https://youtu.be/1zNOrzrwmXo}{Click here for the solution}
  \end{itemize}
\end{enumerate}

\subsection*{Additional Problems:}\label{additional-problems-4}
\addcontentsline{toc}{subsection}{Additional Problems:}

Additional problems that are typically done in class (with video solutions) can be found here:

\begin{itemize}
\tightlist
\item
  \href{https://theelementsmath.github.io/M114/business-economics.html\#cvp-analysis}{CVP Analysis}
\end{itemize}

\section*{Lecture 6: Monopoly}\label{lecture-6-monopoly}
\addcontentsline{toc}{section}{Lecture 6: Monopoly}

\subsection*{Learning Outcomes:}\label{learning-outcomes-5}
\addcontentsline{toc}{subsection}{Learning Outcomes:}

\begin{enumerate}
\def\labelenumi{\arabic{enumi}.}
\tightlist
\item
  Differentiate polynomial functions using standard rules of differentiation.
\item
  Define the characteristics of a monopoly and how it differs from perfect competition.
\item
  Explain how a monopolist determines output and price to maximize profit.
\item
  Interpret graphical representations of monopoly pricing, marginal revenue, and marginal cost.
\item
  Define and apply the profit function: \(\text{Profit}(Q) = TR(Q) - TC(Q)\).
\item
  Use calculus to find the output level that maximizes profit by setting \(\text{Profit}^{\prime}(Q) = 0\).
\item
  Understand and apply the condition \(MR = MC\) for profit maximization.
\end{enumerate}

\subsection*{Review Problems From Last Lecture:}\label{review-problems-from-last-lecture-4}
\addcontentsline{toc}{subsection}{Review Problems From Last Lecture:}

\begin{enumerate}
\def\labelenumi{\arabic{enumi}.}
\tightlist
\item
  The price of a widget is \$35 while the cost per widget is \$20. The fixed costs for the widget factory are \$15,000. Answer the following questions:

  \begin{enumerate}
  \def\labelenumii{\alph{enumii}.}
  \tightlist
  \item
    What is the profit function?
  \item
    What is the contribution margin?
  \item
    What is the break-even volume?
  \item
    What volume is needed for a profit of \$10,000 to be made?
  \item
    If price changes, at what point would you shut down your business?

    \href{https://youtu.be/sF3tnZ02KkU}{Click here for the solution}
  \end{enumerate}
\item
  The profit function for a particular firm is:
  \[
      \text{Profit}(x) = 85x - 10,\!000
      \]
  If the price of the good is \$200 per unit, what are the per unit variable costs?

  \href{https://youtu.be/KBgF9U4clp8}{Click here for the solution}
\item
  A particular company breaks even after selling 500 units. If the company sells 800 units, they make \$6,000.

  \begin{enumerate}
  \def\labelenumii{\alph{enumii}.}
  \tightlist
  \item
    What is the profit function?
  \item
    What is the contribution margin?
  \item
    If the per unit costs are \$50, what is the price of the good?
  \item
    Should these costs change, at what point would the firm consider shutting down?

    \href{https://youtu.be/gxVwF3i6Kr8}{Click here for the solution}
  \end{enumerate}
\end{enumerate}

\subsection*{Lecture Notes:}\label{lecture-notes-5}
\addcontentsline{toc}{subsection}{Lecture Notes:}

Lecture material for this class come from Sections 3.1 and 3.4 and can be found below. This material is considered review material and so it is not covered in depth.

\begin{enumerate}
\def\labelenumi{\arabic{enumi}.}
\tightlist
\item
  \textbf{Video: \href{https://www.youtube.com/watch?v=8Sv6CNuNwqo}{Derivatives of Polynomials}}\\
  The derivative of a polynomial function gives the rate of change or slope at any point.

  \begin{itemize}
  \tightlist
  \item
    Power Rule: If \(f(x) = ax^n\), then \(f'(x) = anx^{n-1}\).
  \item
    Example:\\
    \[
       \text{If } f(x) = 3x^4 - 2x^2 + 5x - 7, \quad \text{then } f'(x) = 12x^3 - 4x + 5
       \]
  \item
    Derivatives help identify turning points, increasing/decreasing intervals, and points of inflection.
  \end{itemize}
\item
  \textbf{Video: \href{https://www.youtube.com/watch?v=y9M6kmgSnw8}{Polynomial Optimization}}

  \begin{itemize}
  \tightlist
  \item
    Optimization involves finding the maximum or minimum values of a polynomial function by identifying critical points.
  \item
    Critical points occur where \(f'(x) = 0\) or is undefined.
  \end{itemize}
\item
  \textbf{Video: \href{https://www.youtube.com/watch?v=146O2tCFBWs}{What is a Monopoly?}}

  \begin{itemize}
  \tightlist
  \item
    A monopoly is a market structure where a single firm is the sole producer of a good or service.
  \item
    Assumptions of a monopoly:

    \begin{itemize}
    \tightlist
    \item
      Single seller with full control over supply.
    \item
      No close substitutes for the product.
    \item
      High barriers to entry (legal, technological, or resource-based).
    \item
      Price maker --- the firm can influence the market price by adjusting output.
    \end{itemize}
  \item
    Monopolies typically face a downward-sloping demand curve.
  \end{itemize}
\item
  \textbf{Video: \href{https://www.youtube.com/watch?v=Qk8QdpJovfs}{Profit Maximization}}

  \begin{itemize}
  \tightlist
  \item
    A monopoly maximizes profit where marginal revenue (MR) equals marginal cost (MC) or when the derivative of the profit function is 0.
  \item
    Since the monopolist faces a downward-sloping demand curve, \(MR < P\).
  \item
    Profit-maximizing output \(Q^*\) is found where:
    \[
       MR(Q^*) = MC(Q^*)
       \] or where
    \[
       \text{Profit}^{\prime} = 0
       \]\\
  \item
    Price is determined by plugging \(Q^*\) into the demand function.
  \item
    For linear demand, profit is calculated as:
    \[
       \text{Profit} = (P - VC) \times Q^* - FC
       \]
  \end{itemize}
\end{enumerate}

\subsection*{Lecture Problems:}\label{lecture-problems-5}
\addcontentsline{toc}{subsection}{Lecture Problems:}

\begin{enumerate}
\def\labelenumi{\arabic{enumi}.}
\tightlist
\item
  Derivatives of Polynomials

  \begin{itemize}
  \tightlist
  \item
    Find the derivative of the polynomial function:
    \[
    f(x) = 5x^4 - 3x^3 + 2x - 7
    \]
  \item
    Determine the slope of the function \(f(x) = 4x^3 - x + 6\) at \(x = 2\).
  \item
    Explain the significance of the derivative in understanding the behavior of polynomial functions.

    \href{https://youtu.be/C5jkO62wuys}{Click here for the solution}
  \end{itemize}
\item
  Monopolies

  \begin{itemize}
  \tightlist
  \item
    Define a monopoly and list at least three assumptions or characteristics of a monopolistic market.
  \item
    Explain why a monopolist is called a ``price maker.''
  \item
    Discuss the implications of barriers to entry in a monopolistic market.

    \href{https://youtu.be/MNneiFzlWIM}{Click here for the solution}
  \end{itemize}
\item
  Profit Maximization in a Monopoly

  \begin{itemize}
  \tightlist
  \item
    Explain the condition \(MR = MC\) in the context of monopoly profit maximization.
  \item
    Given the demand function \(P = 100 - 2Q\) and total cost function \(TC = 20Q + 100\), find:

    \begin{itemize}
    \tightlist
    \item
      The profit function.
    \item
      The derivative of the profit function.
    \item
      The profit-maximizing output level \(Q^*\).
    \item
      The corresponding price \(P^*\).
    \item
      The maximum profit.
    \end{itemize}
  \item
    Why is marginal revenue less than price for a monopolist?

    \href{hhttps://youtu.be/g207cVSDu6Q}{Click here for the solution}
  \end{itemize}
\item
  A monopolist faces the following demand and total cost functions:
  \[
      \text{Demand: } P = 1200 - 40Q \quad \text{and} \quad \text{Total Cost: } TC = 200Q + 1000
      \]

  \begin{itemize}
  \tightlist
  \item
    Write the total revenue function \(TR(Q)\).
  \item
    Find the profit function.
  \item
    Find the derivative of the profit function.
  \item
    Find the profit-maximizing quantity \(Q^*\).
  \item
    Determine the price \(P^*\) the firm should charge.
  \item
    Calculate the maximum profit.

    \href{https://youtu.be/AZx5xSc00rI}{Click here for the solution}
  \end{itemize}
\end{enumerate}

\subsection*{Additional Problems:}\label{additional-problems-5}
\addcontentsline{toc}{subsection}{Additional Problems:}

Additional problems that are typically done in class (with video solutions) can be found here:

\begin{itemize}
\tightlist
\item
  \href{https://theelementsmath.github.io/M114/business-economics.html\#monopoly}{Monopolies}
\end{itemize}

\part{Simple and Compound Interest}\label{part-simple-and-compound-interest}

\chapter{Simple Interest}\label{simple-interest}

\section*{Lecture 7: Simple Interest}\label{lecture-7-simple-interest}
\addcontentsline{toc}{section}{Lecture 7: Simple Interest}

\subsection*{Learning Outcomes:}\label{learning-outcomes-6}
\addcontentsline{toc}{subsection}{Learning Outcomes:}

\begin{enumerate}
\def\labelenumi{\arabic{enumi}.}
\tightlist
\item
  Define the terms principal (\(P\)), rate (\(r\)), time (\(t\)), and interest (\(I\)) in the context of simple interest.
\item
  State and interpret the simple interest formula: \(I = Prt\).
\item
  Solve for the missing variable (e.g., \(P\), \(r\), or \(t\)) in the formula when the other three are known.
\item
  Convert between different time units (e.g., months to years) to ensure consistency in the formula.
\item
  Apply the simple interest formula to solve real-world financial problems.
\end{enumerate}

\subsection*{Review Problems From Last Lecture:}\label{review-problems-from-last-lecture-5}
\addcontentsline{toc}{subsection}{Review Problems From Last Lecture:}

\begin{enumerate}
\def\labelenumi{\arabic{enumi}.}
\tightlist
\item
  A monopolist does some market research and finds that he has a profit function given by:
  \[
      \text{Profit}(x) = -5x^2 + 6300x - 1,\!500,\!000
      \]
  What is the profit-maximizing level of output and the maximum profit the monopolist can realize?

  \href{https://youtu.be/M4wUBGg8I7o}{Click here for the solution}
\item
  A monopolist has done some market research and determined that the demand curve for his good is given by:
  \[
      P = -15x + 15,\!010
      \]
  Fixed costs for the industry amount to roughly \$1,000,000 while per unit costs are \$10 per unit.

  \begin{itemize}
  \tightlist
  \item
    What is the profit-maximizing level of output?
  \item
    How many units should the monopolist produce?\\
  \item
    At what price should the monopolist sell the good?

    \href{https://youtu.be/9tuhKbFdCow}{Click here for the solution}
  \end{itemize}
\end{enumerate}

\subsection*{Lecture Notes:}\label{lecture-notes-6}
\addcontentsline{toc}{subsection}{Lecture Notes:}

Lecture material for this class come from Section 4.1 - 4.3 and can be found below. This material is considered review material and so it is not covered in depth.

\begin{enumerate}
\def\labelenumi{\arabic{enumi}.}
\tightlist
\item
  \textbf{Video: \href{https://youtu.be/zxGUXpDlWgg}{Simple Interest}}\\
  Simple interest is the interest calculated only on the principal amount, not on any interest previously earned. Simple interest is calculated using the formula \[I = Prt,\] where:

  \begin{itemize}
  \tightlist
  \item
    \(I\) = interest earned (\$)
  \item
    \(P\) = principal (initial investment or loan amount)
  \item
    \(r\) = annual interest rate (in decimal form)
  \item
    \(t\) = time (in years)
  \end{itemize}
\end{enumerate}

\subsection*{Lecture Problems:}\label{lecture-problems-6}
\addcontentsline{toc}{subsection}{Lecture Problems:}

\begin{enumerate}
\def\labelenumi{\arabic{enumi}.}
\tightlist
\item
  John invests \$2,000 at an annual simple interest rate of 5\% for 4 years.\\
  How much interest will he earn?

  \href{https://youtu.be/BuhohsmwhBI}{Click here for the solution}
\item
  A loan of \$1,500 earns \$225 in simple interest at an annual rate of 6\%. For how many years was the money borrowed?

  \href{https://youtu.be/lRpx_TdHt-w}{Click here for the solution}
\item
  If \$360 is earned in 3 years at a simple interest rate of 4\%. What was the original principal?

  \href{https://youtu.be/crzQqex-stg}{Click here for the solution}
\item
  Sarah earned \$450 in interest after investing \$2,500 for 3 years. What was the annual simple interest rate?

  \href{https://youtu.be/vG7-oWsfi7g}{Click here for the solution}
\item
  A student deposits \$1,800 in a savings account at a simple interest rate of 3.5\% for 2 years. What will be the total amount in the account at the end of 2 years?

  \href{https://youtu.be/_ULZWLLe1v8}{Click here for the solution}
\item
  A borrower takes out a loan of \$3,000 at 10\% simple interest. How much interest is owed after 9 months?

  \href{https://youtu.be/cr8Rm5gLASA}{Click here for the solution}
\end{enumerate}

\subsection*{Additional Problems:}\label{additional-problems-6}
\addcontentsline{toc}{subsection}{Additional Problems:}

Additional problems that are typically done in class (with video solutions) can be found here:

\begin{itemize}
\tightlist
\item
  \href{https://theelementsmath.github.io/M114/simple-interest.html\#simple-interest-1}{Simple Interest}
\end{itemize}

\section*{Lecture 8: Present and Future Value}\label{lecture-8-present-and-future-value}
\addcontentsline{toc}{section}{Lecture 8: Present and Future Value}

\subsection*{Learning Outcomes:}\label{learning-outcomes-7}
\addcontentsline{toc}{subsection}{Learning Outcomes:}

\begin{enumerate}
\def\labelenumi{\arabic{enumi}.}
\tightlist
\item
  Define key financial concepts related to simple interest, including principal (\(P\)), rate (\(r\)), time (\(t\)), and \{future value\} (\(FV\)).
\item
  Calculate the \{Future Value (FV)\} using the formula:
  \[
   FV = P(1 + rt)
   \]
\item
  Determine the \{present value (PV)\} using the formula:
  \[
   P = \frac{FV}{1 + rt}
   \]
\item
  Solve for the \{interest rate (r)\} when \(FV\), \(P\), and \(t\) are known:
  \[
   r = \frac{I}{P t}
   \]
\item
  Solve for \{time (t)\} when \(FV\), \(P\), and \(r\) are known:
  \[
   t = \frac{I}{P r}
   \]
\item
  Interpret and explain the results of simple interest calculations in practical financial contexts.
\end{enumerate}

\subsection*{Review Problems From Last Lecture:}\label{review-problems-from-last-lecture-6}
\addcontentsline{toc}{subsection}{Review Problems From Last Lecture:}

\begin{enumerate}
\def\labelenumi{\arabic{enumi}.}
\tightlist
\item
  You deposit \$15,000 into an account earning 4.75\% simple interest. How much interest is earned after 4, 8, and 12 months?

  \href{https://youtu.be/HuDASjD4nMM}{Click here for the solution}
\item
  How long will it take for an investment of \$900 to earn \$150 in interest at a rate of 4.25\% simple interest? Find the time in years and months.

  \href{https://youtu.be/qlWcS-9gIA0}{Click here for the solution}
\item
  What is the value of a \$750 deposit after 15 months if the deposit earns 6.5\% simple interest? Find the total amount (principal + interest).

  \href{https://youtu.be/TbWUIo3dGTk}{Click here for the solution}
\item
  An investment of \$1,200 grew to \$1,275 between June 13, 2009 and August 25, 2010. What annual rate of simple interest was realized?

  \href{https://youtu.be/fTSXKSAGclc}{Click here for the solution}
\item
  You make 3 deposits of \$500 into an account earning 3.8\% simple interest. The first is in January, the second in April, and the final deposit in August. How much money will be in the account in November? (Assume all deposits are made at the beginning of each month.)

  \href{https://youtu.be/UOrxaQ4701U}{Click here for the solution}
\end{enumerate}

\subsection*{Lecture Notes:}\label{lecture-notes-7}
\addcontentsline{toc}{subsection}{Lecture Notes:}

Lecture material for this class come from Sections 4.4 -- 4.5 and can be found below. This material is considered review material and so it is not covered in depth.

\begin{enumerate}
\def\labelenumi{\arabic{enumi}.}
\tightlist
\item
  \textbf{Video: \href{https://youtu.be/rO37XyVQjW8}{Present and Future Value}}\\
  The \textbf{future value} of a simple interest investment is the total amount an investment will grow to, including interest. Future value is calculated using the formula \[FV = P(1 + rt),\] where:

  \begin{itemize}
  \tightlist
  \item
    \(FV\) = future value (amount after interest)
  \item
    \(P\) = principal (initial investment)
  \item
    \(r\) = annual interest rate (as a decimal)
  \item
    \(t\) = time in years
  \end{itemize}
\item
  The \textbf{present value} of a simple interest investments the amount that must be invested today to reach a desired future value. Present value is calculated using the formula \[ P = \frac{FV}{1 + rt}.\]
\item
  Rearranging the formula, we can find the term or rate.

  \begin{itemize}
  \tightlist
  \item
    Rate: \(r = \dfrac{FV - P}{Pt} = \dfrac{I}{Pt}\)
  \item
    Time: \(t = \dfrac{FV - P}{Pr} = \dfrac{I}{Pr}\)
  \end{itemize}
\end{enumerate}

\subsection*{Lecture Problems:}\label{lecture-problems-7}
\addcontentsline{toc}{subsection}{Lecture Problems:}

\begin{enumerate}
\def\labelenumi{\arabic{enumi}.}
\tightlist
\item
  You invest \$4,000 at an annual simple interest rate of 5.2\% for 3 years. What is the future value of the investment?

  \href{https://youtu.be/ooJiLig1RI8}{Click here for the solution}
\item
  You want to have \$6,500 in 4 years. The account earns 4.75\% simple interest annually. How much do you need to invest today?

  \href{https://youtu.be/qZ7aX93LZbE}{Click here for the solution}
\item
  A \$2,000 investment grows to \$2,260 in 2 years. What simple interest rate was applied annually?

  \href{https://youtu.be/QhSdkW94xBM}{Click here for the solution}
\item
  An investment of \$1,500 grows to \$1,725 at a simple interest rate of 5\%. How long was the money invested?

  \href{https://youtu.be/fTSXKSAGclc}{Click here for the solution}
\item
  A \$1,200 deposit grows to \$1,260 in 9 months. What was the annual simple interest rate?

  \href{https://youtu.be/R5s-Mz-ZkIc}{Click here for the solution}
\item
  An investment of \$800 earns \$88 in simple interest at an annual rate of 5.5\%. How long was the money invested? Express your answer in months.

  \href{https://youtu.be/GwhT6TUKA1Y}{Click here for the solution}
\end{enumerate}

\subsection*{Additional Problems:}\label{additional-problems-7}
\addcontentsline{toc}{subsection}{Additional Problems:}

Additional problems that are typically done in class (with video solutions) can be found here:

\begin{itemize}
\item
  \href{https://theelementsmath.github.io/M114/simple-interest.html\#present-and-future-value}{Present and Future Value}
\item
  \href{https://theelementsmath.github.io/M114/simple-interest.html\#rate-of-interest-and-term-of-investments}{Rate of Interest and Term of Investments}
\end{itemize}

\section*{Lecture 9: Equivalent Payments}\label{lecture-9-equivalent-payments}
\addcontentsline{toc}{section}{Lecture 9: Equivalent Payments}

\subsection*{Learning Outcomes:}\label{learning-outcomes-8}
\addcontentsline{toc}{subsection}{Learning Outcomes:}

\begin{enumerate}
\def\labelenumi{\arabic{enumi}.}
\tightlist
\item
  Define the concept of equivalent payments under simple interest.
\item
  Determine the time value of money using equivalent payment comparisons.
\item
  Calculate an equivalent payment at a different time using simple interest formulas.
\item
  Identify and describe payment streams in the context of simple interest.
\item
  Calculate the future or present value of multiple payments made at different times.
\item
  Evaluate total interest earned or paid on a series of timed payments.
\item
  Apply timeline and tabular methods to organize and solve payment stream problems.
\item
  Understand how interest rate changes affect simple interest calculations.
\item
  Calculate interest over multiple time periods with different interest rates.
\end{enumerate}

\subsection*{Review Problems From Last Lecture:}\label{review-problems-from-last-lecture-7}
\addcontentsline{toc}{subsection}{Review Problems From Last Lecture:}

\begin{enumerate}
\def\labelenumi{\arabic{enumi}.}
\tightlist
\item
  An investment of \$3,000 is made at a simple interest rate of 4.5\% per annum for 2 years. What is the future value of the investment?

  \href{https://youtu.be/7qmcWXW8fos}{Click here for the solution}
\item
  A principal of \$2,400 is invested at a rate of 5.2\% simple interest for 15 months. What is the future value at the end of the term?

  \href{https://youtu.be/OxLt9Az2QA8}{Click here for the solution}
\item
  A deposit of \$1,000 grows to \$1,040 in 8 months. What is the annual simple interest rate?

  \href{hhttps://youtu.be/D1t4ZLWUpVI}{Click here for the solution}
\item
  How long will it take for \$700 to earn \$98 in interest at 7\% annual simple interest? Give your answer in years.

  \href{https://youtu.be/inwp5jWlTSo}{Click here for the solution}
\end{enumerate}

\subsection*{Lecture Notes:}\label{lecture-notes-8}
\addcontentsline{toc}{subsection}{Lecture Notes:}

Lecture material for this class come from Sections 1.1 -- 1.5 and can be found below. This material is considered review material and so it is not covered in depth.

\begin{enumerate}
\def\labelenumi{\arabic{enumi}.}
\tightlist
\item
  \textbf{Video: \href{https://youtu.be/eCy0_cPgobc}{Payment Streams}}\\
  A \textbf{payment stream} is a series of individual payments made at different points in time.

  \begin{itemize}
  \tightlist
  \item
    Each payment earns interest independently from the time it is deposited until the calculation date (e.g., present or future).
  \item
    To find the \textbf{total future value} or \textbf{present value} of a payment stream:

    \begin{itemize}
    \tightlist
    \item
      Calculate the future/present value of each payment separately.
    \item
      Add the individual values to find the total.
    \end{itemize}
  \item
    Timeline diagrams are helpful for visualizing and organizing the timing of each payment.
  \end{itemize}
\item
  \textbf{Video: \href{https://youtu.be/0v4t_Xp7tGo}{Equivalent Payments}}\\
  Two payments are said to be \textbf{equivalent} if they have the same value at the same point in time, considering the effect of interest.

  \begin{itemize}
  \tightlist
  \item
    To compare payments made at different times:

    \begin{itemize}
    \tightlist
    \item
      Bring both payments to the same date (using present or future value formulas).
    \item
      Use \(FV = P(1 + rt)\) or \(P = \frac{FV}{1 + rt}\).
    \end{itemize}
  \item
    Equivalent payments are used to replace one payment with another while maintaining the same economic value.
  \end{itemize}
\item
  \textbf{Video: \href{https://youtu.be/ooUss4RfLHI}{Systems of Equations I}}\\
  Changing interest rates: In real-world scenarios, the interest rate may change during the investment or loan period.

  \begin{itemize}
  \tightlist
  \item
    For simple interest, calculate interest in segments:

    \begin{itemize}
    \tightlist
    \item
      Break the time into periods where each rate applies.
    \item
      Apply simple interest separately for each period.
    \item
      Sum the interest amounts to get the total interest.
    \end{itemize}
  \item
    Formula for each segment: \(I = Prt\)
  \item
    Final amount:
    \[
     FV = P + I_1 + I_2 + \ldots + I_n
     \]
  \end{itemize}
\end{enumerate}

\subsection*{Lecture Problems:}\label{lecture-problems-8}
\addcontentsline{toc}{subsection}{Lecture Problems:}

\begin{enumerate}
\def\labelenumi{\arabic{enumi}.}
\tightlist
\item
  You make three deposits: \$500 in January, \$600 in April, and \$800 in August. If the account earns 4\% simple interest annually, What is the total value of the account at the end of December? (Assume deposits are made at the beginning of each month.)

  \href{b}{Click here for the solution}
\item
  A business makes payments of \$1,000, \$1,200, and \$1,500 at 6-month intervals into a fund earning 3.5\% simple interest. Find the total future value of all payments 18 months after the first deposit.

  \href{https://youtu.be/_c-59B-eq24}{Click here for the solution}
\item
  A loan of \$2,000 is to be repaid in two equal payments: one in 6 months and one in 12 months. If interest is 5\% simple interest annually,\\
  What should each payment be to fully repay the loan?

  \href{https://youtu.be/QK3yk1NInE4}{Click here for the solution}
\item
  A payment of \$1,200 is due in 10 months. What is the equivalent value of that payment if it is made 4 months from now instead, assuming 4.5\% simple interest?

  \href{https://youtu.be/Ggslh8jfGrk}{Click here for the solution}
\item
  A company can either pay \$2,000 now or \$2,150 in 5 months. Which option is better if the interest rate is 7\% simple interest annually?

  \href{https://youtu.be/e_qfCFdWR4A}{Click here for the solution}
\item
  An investor deposits \$10,000. The interest rate is 4\% for the first 3 months, 4.5\% for the next 3 months, and 5\% for the next 6 months. Find the total interest earned at the end of 1 year.

  \href{https://youtu.be/PunaGnanvsA}{Click here for the solution}
\end{enumerate}

\subsection*{Additional Problems:}\label{additional-problems-8}
\addcontentsline{toc}{subsection}{Additional Problems:}

Additional problems that are typically done in class (with video solutions) can be found here:

\begin{itemize}
\item
  \href{https://theelementsmath.github.io/M114/simple-interest.html\#equivalent-payments}{Equivalent Payments}
\item
  \href{https://theelementsmath.github.io/M114/simple-interest.html\#changing-interest-rates}{Changing Interest Rates}
\end{itemize}

\chapter{Compound Interest}\label{compound-interest}

\section*{Lecture 10: Present and Future Value}\label{lecture-10-present-and-future-value}
\addcontentsline{toc}{section}{Lecture 10: Present and Future Value}

\subsection*{Learning Outcomes:}\label{learning-outcomes-9}
\addcontentsline{toc}{subsection}{Learning Outcomes:}

\begin{enumerate}
\def\labelenumi{\arabic{enumi}.}
\tightlist
\item
  Define and distinguish between \textbf{present value} and \textbf{future value} in the context of compound interest.
\item
  Use the compound interest formula:
  \[
   FV = PV\left(1 + i\right)^{n}
   \]
  to calculate the future value of an investment, where

  \begin{itemize}
  \tightlist
  \item
    \(FV\): future value
  \item
    \(PV\): present value (principal)
  \item
    \(i\): periodic interest rate (per compound period)
  \item
    \(n\): number of compounding periods for the investment
  \end{itemize}
\item
  Use the rearranged formula:
  \[
   PV = \dfrac{FV}{\left(1 + i\right)^{n}}
   \]
  to determine the present value required to reach a given future value.
\item
  Solve problems involving compound interest with annual, semi-annual, quarterly, monthly, and daily compounding frequencies.
\item
  Apply compound interest formulas to real-world financial scenarios such as savings accounts, investment growth, and loan balances.
\end{enumerate}

\subsection*{Review Problems From Last Lecture:}\label{review-problems-from-last-lecture-8}
\addcontentsline{toc}{subsection}{Review Problems From Last Lecture:}

\begin{enumerate}
\def\labelenumi{\arabic{enumi}.}
\tightlist
\item
  What is the fair market value today of 3 payments of \$1,000. The first payment is to be made 150 days from now, the second 225 days from now and the final one 300 days from now. Use 8.4\% simple interest for all calculations.

  \href{https://youtu.be/SQhAIwiVOwc}{Click here for the solution}
\item
  You deposit \$2,000 now, \$1,500 in 6 months, and \$1,000 in 1 year. If the interest rate is 4.2\% per annum simple interest, What is the total future value of all deposits at the end of 18 months?

  \href{https://youtu.be/2wadK5_GH5E}{Click here for the solution}
\item
  A \$6,000 investment earns 3.5\% simple interest for 4 months, then 5\% for the next 8 months. Calculate the total interest earned and the final value after 1 year.

  \href{https://youtu.be/xhKMpIKqCFA}{Click here for the solution}
\item
  A payment of \$950 is due in 9 months. You want to replace it with an equivalent payment made in 2 months. What amount would be equivalent if the interest rate is 5.5\% per annum simple interest?

  \href{https://youtu.be/zzLoWKDcHYc}{Click here for the solution}
\end{enumerate}

\subsection*{Lecture Notes:}\label{lecture-notes-9}
\addcontentsline{toc}{subsection}{Lecture Notes:}

Lecture material for this class come from Sections 5.1 -- 5.2 and can be found below. This material is considered review material and so it is not covered in depth.

\begin{enumerate}
\def\labelenumi{\arabic{enumi}.}
\tightlist
\item
  \textbf{Video: \href{https://youtu.be/y99qpJXtx3Q}{The Basics of CI}}\\
  The total number of compounding periods is \(n = m \cdot t\). The value of \(m\) determines the compounding frequency.

  \begin{itemize}
  \tightlist
  \item
    Annually: \(m = 1\)
  \item
    Semi-annually: \(m = 2\)
  \item
    Quarterly: \(m = 4\)
  \item
    Monthly: \(m = 12\)
  \item
    Daily: \(m = 365\)
  \end{itemize}
\item
  \textbf{Video: \href{https://youtu.be/YyV8hybwRT0}{Compound Interest FV}}\\
  \textbf{Video: \href{https://youtu.be/xu3JoFjM82w}{Compound Interest PV}}\\
  Compound interest\} is interest calculated on the initial principal \textbf{and} also on the accumulated interest from previous periods. It is calculated using the formula: \[
  FV = PV(1 + i)^n
  \]
  Where:
  \textbackslash begin\{itemize\}

  \begin{itemize}
  \tightlist
  \item
    \(FV\): Future Value (total amount after interest)
  \item
    \(PV\): Present Value (initial investment or loan)
  \item
    \(i\): Interest rate per compounding period \(\left( i = \frac{r}{m} \right)\)
  \item
    \(n\): Total number of compounding periods \(\left( n = m \cdot t \right)\)
  \item
    \(r\): Annual nominal interest rate (as a decimal)
  \item
    \(m\): Number of compounding periods per year
  \item
    \(t\): Time in years
  \end{itemize}
\end{enumerate}

\subsection*{Lecture Problems:}\label{lecture-problems-9}
\addcontentsline{toc}{subsection}{Lecture Problems:}

\begin{enumerate}
\def\labelenumi{\arabic{enumi}.}
\tightlist
\item
  You invest \$2,500 at an annual interest rate of 5.2\% compounded monthly for 3 years. What is the future value of the investment?

  \href{https://youtu.be/5w9hc7thqwE}{Click here for the solution}
\item
  \$1,800 is deposited into an account earning 4.5\% annual interest compounded quarterly for 5 years. Find the amount in the account at the end of the term.

  \href{https://youtu.be/tLfjhf_mshg}{Click here for the solution}
\item
  A deposit of \$6,000 is made into a savings account with 3.9\% interest compounded annually for 10 years. What will the investment be worth at maturity?

  \href{https://youtu.be/9CMaL7HUPEs}{Click here for the solution}
\item
  You want to have \$10,000 in 4 years. The account offers 6\% interest compounded semi-annually. How much should you invest today?

  \href{https://youtu.be/p5kNaZKyv7g}{Click here for the solution}
\item
  How much should be invested now to grow to \$7,500 in 3 years at 5.25\% interest compounded monthly?

  \href{https://youtu.be/d2MkmbyQmG8}{Click here for the solution}
\item
  A future amount of \$12,000 is needed in 6 years. If the account pays 4.8\% compounded quarterly, what is the present value of the investment?

  \href{https://youtu.be/MmyATTVMOK4}{Click here for the solution}
\end{enumerate}

\subsection*{Additional Problems:}\label{additional-problems-9}
\addcontentsline{toc}{subsection}{Additional Problems:}

Additional problems that are typically done in class (with video solutions) can be found here:

\begin{itemize}
\tightlist
\item
  \href{https://theelementsmath.github.io/M114/compound-interest.html\#present-and-future-values}{Present and Future Value (Compound Interest)}
\end{itemize}

\section*{Lecture 11: Compound Interest - Rate and Term}\label{lecture-11-compound-interest---rate-and-term}
\addcontentsline{toc}{section}{Lecture 11: Compound Interest - Rate and Term}

\subsection*{Learning Outcomes:}\label{learning-outcomes-10}
\addcontentsline{toc}{subsection}{Learning Outcomes:}

\begin{enumerate}
\def\labelenumi{\arabic{enumi}.}
\item
  Determine the \textbf{number of compounding periods} (\(n\)) using the formula:
  \[
   n = \dfrac{\ln(FV/PV)}{\ln(1+i)}.
   \]
\item
  Identify and calculate the \textbf{term} (\(t\)) of an investment or loan in years, months, or other units,
  \[
   t = \frac{n}{m},
   \]
  where \(m\) is the number of compounding periods per year.
\item
  Calculate the \textbf{periodic interest rate}:
  \[
   i = \left(\dfrac{FV}{PV} \right)^{1/n} - 1.
   \]
  where \(r\) is the annual nominal interest rate.
\item
  Calculate the \textbf{nominal interest rate}:
  \[
   r = i \times m,
   \]
  where \(r\) is the annual nominal interest rate.
\end{enumerate}

\subsection*{Review Problems From Last Lecture:}\label{review-problems-from-last-lecture-9}
\addcontentsline{toc}{subsection}{Review Problems From Last Lecture:}

\begin{enumerate}
\def\labelenumi{\arabic{enumi}.}
\tightlist
\item
  You invest \$4,000 at an annual interest rate of 5\% compounded monthly for 3 years. What is the future value of the investment?

  \href{https://youtu.be/Csdbw7AIGSA}{Click here for the solution}
\item
  You invest \$7,200 in an account that earns 4.25\% compounded annually for 8 years. How much will the investment be worth at the end of the term?

  \href{https://youtu.be/4vo1QE6fmXE}{Click here for the solution}
\item
  You want to have \$10,000 in 4 years. If the interest rate is 5.5\% compounded semi-annually, how much should you invest today?

  \href{https://youtu.be/AZg71adyIfo}{Click here for the solution}
\item
  A savings goal of \$8,000 must be met in 3 years. If the bank offers 4.2\% compounded monthly, what is the required initial deposit?

  \href{https://youtu.be/hCXRx8LGUe4}{Click here for the solution}
\end{enumerate}

\subsection*{Lecture Notes:}\label{lecture-notes-10}
\addcontentsline{toc}{subsection}{Lecture Notes:}

Lecture material for this class come from Sections 1.1 -- 1.5 and can be found below. This material is considered review material and so it is not covered in depth.

\begin{enumerate}
\def\labelenumi{\arabic{enumi}.}
\tightlist
\item
  \textbf{Video: \href{https://youtu.be/n2Z1aLFtaiE}{Equivalent Interest Rate}}\\
  The \textbf{annual nominal interest rate} is the stated annual interest rate. The \textbf{periodic interest rate} is the interest rate applied each compounding period. Examples include:

  \begin{itemize}
  \tightlist
  \item
    Annual compounding: \(i = r\)
  \item
    Semi-annual: \(i = \frac{r}{2}\)
  \item
    Quarterly: \(i = \frac{r}{4}\)
  \item
    Monthly: \(i = \frac{r}{12}\)
  \item
    Daily: \(i = \frac{r}{365}\)
    \[
     i = \left(\dfrac{FV}{PV} \right)^{1/n} - 1.
     \]
  \end{itemize}
\item
  \textbf{Video: \href{https://youtu.be/rfa1WBzjMaY}{Term of Investment}}\\
  The \textbf{term of the investment} is how long the money is invested or borrowed. The \textbf{number of compounding periods} is the number of times interest is applied over the entire term.\\
  \[
   n = \dfrac{\ln(FV/PV)}{\ln(1+i)}.
   \]
\end{enumerate}

\subsection*{Lecture Problems:}\label{lecture-problems-10}
\addcontentsline{toc}{subsection}{Lecture Problems:}

\begin{enumerate}
\def\labelenumi{\arabic{enumi}.}
\tightlist
\item
  \$3,000 grows to \$3,828.81 in a bank account earning 6\% interest compounded annually. How long was the money invested?

  \href{https://youtu.be/ala74SWNy-Y}{Click here for the solution}
\item
  A deposit of \$2,500 grows to \$3,262.78 at an interest rate of 4.5\% compounded quarterly. How many years did it take to reach that value?

  \href{https://youtu.be/FTwVwzi5Nf0}{Click here for the solution}
\item
  \$4,000 is invested for 4 years and grows to \$5,041.60. The interest is compounded annually. What is the annual nominal interest rate?

  \href{https://youtu.be/3plLnxnEtHU}{Click here for the solution}
\item
  An investment of \$6,500 grows to \$8,312.34 over 6 years, compounded monthly. Find the annual interest rate.

  \href{https://youtu.be/fnOVWqeE4gI}{Click here for the solution}
\end{enumerate}

\subsection*{Additional Problems:}\label{additional-problems-10}
\addcontentsline{toc}{subsection}{Additional Problems:}

Additional problems that are typically done in class (with video solutions) can be found here:

\begin{itemize}
\tightlist
\item
  \href{https://theelementsmath.github.io/M114/compound-interest.html\#calculating-the-periodic-interest-rate-and-number-of-compounding-periods}{Calculating the Periodic Interest Rate and Number of Compounding Periods}
\end{itemize}

\section*{Lecture 12: Payment Streams and Equivalent Payments}\label{lecture-12-payment-streams-and-equivalent-payments}
\addcontentsline{toc}{section}{Lecture 12: Payment Streams and Equivalent Payments}

\subsection*{Learning Outcomes:}\label{learning-outcomes-11}
\addcontentsline{toc}{subsection}{Learning Outcomes:}

\begin{enumerate}
\def\labelenumi{\arabic{enumi}.}
\tightlist
\item
  Define a payment stream and describe how it differs from a single lump sum investment.
\item
  Calculate the future value and present value of a stream of compound interest payments.
\item
  Define the concept of equivalent payments and when it applies in financial contexts.
\item
  Determine the time value equivalence of payments made at different times and under different interest rates.
\item
  Convert a series of unequal payments into a single equivalent payment at a specified point in time.
\item
  Solve problems involving multiple interest rates applied over different time intervals.
\item
  Use piecewise calculations to find the total future or present value when rates change partway through an investment or loan.
\item
  Define continuously compounded interest and compare it to standard compound interest.
\item
  Use the formula:
  \[
   FV = PV \cdot e^{rt}
   \]
  to calculate future value with continuous compounding.
\end{enumerate}

\subsection*{Review Problems From Last Lecture:}\label{review-problems-from-last-lecture-10}
\addcontentsline{toc}{subsection}{Review Problems From Last Lecture:}

\begin{enumerate}
\def\labelenumi{\arabic{enumi}.}
\tightlist
\item
  An investment of \$3,000 grows to \$3,915.79 in 4 years with quarterly compounding. What annual interest rate was earned?

  \href{https://youtu.be/vQc-7wXZqlk}{Click here for the solution}
\item
  \$5,000 grows to \$6,734.29 at 6.5\% annual interest compounded monthly. How long was the money invested?

  \href{https://youtu.be/M5UPZ7Sv0Hg}{Click here for the solution}
\item
  A deposit of \$1,200 increases to \$1,579.35 over 5 years with monthly compounding. Find the annual nominal interest rate

  \href{https://youtu.be/SNO9GQOu4Bs}{Click here for the solution}
\item
  A \$1,500 investment becomes \$2,100 at 5\% compounded quarterly. Find the number of years required.

  \href{https://youtu.be/sesqd-K7qTM}{Click here for the solution}
\end{enumerate}

\subsection*{Lecture Notes:}\label{lecture-notes-11}
\addcontentsline{toc}{subsection}{Lecture Notes:}

Lecture material for this class come from Sections 5.4 -- 5.7 and can be found below. This material is considered review material and so it is not covered in depth.

\begin{enumerate}
\def\labelenumi{\arabic{enumi}.}
\tightlist
\item
  \textbf{Video: \href{https://youtu.be/HLUDkVUHNwM}{Payment Streams}}\\
  A \textbf{payment stream} is a series of cash flows (payments or deposits) made at regular or irregular intervals over time. These are often used in loan payments, investments, or savings plans.

  \begin{itemize}
  \tightlist
  \item
    Each payment may earn interest depending on when it is made and when it is evaluated.
  \item
    Time diagrams are useful for visualizing and organizing payment streams.
  \item
    Payment streams can be evaluated using the concept of future value or present value by summing each individual amount's value at the chosen focal date.
  \end{itemize}
\item
  \textbf{Video: \href{https://youtu.be/HLUDkVUHNwM}{Equivalent Payments}}\\
  \textbf{Equivalent payments} are payments made at different times that have the same value when brought to a common point in time using interest calculations.

  \begin{itemize}
  \tightlist
  \item
    Payments are equivalent if their present or future values are equal at a given focal date.
  \item
    Used to simplify or compare payment schedules.
    Steps:
  \item
    Choose a focal date.
  \item
    Convert each payment to its value at the focal date using simple or compound interest.
  \item
    Equate values and solve for unknowns (e.g., payment amount, date, interest rate).
  \end{itemize}
\item
  In some financial problems, the interest rate changes at specific intervals during the term of the investment or loan.\\
  Approach:

  \begin{itemize}
  \tightlist
  \item
    Split the time period into segments where each rate applies.
  \item
    Apply the interest formula separately for each segment.
  \item
    Use the output of one period as the input (principal) for the next.
    Formula:
    \[
    FV = PV \cdot (1 + i_1)^{n_1} \cdot (1 + i_2)^{n_2} \cdots (1 + i_k)^{n_k}
    \]
    Where:
  \item
    \(i_k\) is the periodic interest rate during segment \(k\)
  \item
    \(n_k\) is the number of periods for rate \(i_k\)
  \end{itemize}
\item
  The \textcolor{red}{Continuous compounding} assumes interest is being added at every possible instant. It represents the mathematical limit of compounding frequency.
  Formula:
  \[
  FV = PV \cdot e^{rt}
  \]
  Where:

  \begin{itemize}
  \tightlist
  \item
    \(e \approx 2.71828\) is Euler's number
  \item
    \(r\) is the annual nominal interest rate (in decimal)
  \item
    \(t\) is time in years
  \end{itemize}
\end{enumerate}

\subsection*{Lecture Problems:}\label{lecture-problems-11}
\addcontentsline{toc}{subsection}{Lecture Problems:}

\begin{enumerate}
\def\labelenumi{\arabic{enumi}.}
\tightlist
\item
  Three payments of \$800 are made at the end of years 1, 2, and 3 into an account earning 5\% compound interest annually. What is the total value of the payment stream at the end of year 3?

  \href{https://youtu.be/Tqoxyk1YPeo}{Click here for the solution}
\item
  What single payment today is equivalent to two future payments of \$600 in 1 year and \$700 in 2 years, assuming 6\% annual compound interest?

  \href{https://youtu.be/heaWuNOF9Wk}{Click here for the solution}
\item
  A business wants to replace two payments of \$2,000 due at the end of years 2 and 4 with a single equivalent payment at the end of year 3. If the interest rate is 5\% compounded annually, what should the equivalent payment be?

  \href{https://youtu.be/Q_GTSXLL-9c}{Click here for the solution}
\item
  An investment of \$5,000 earns 4\% interest compounded annually for the first 2 years and then 6\% compounded annually for the next 3 years. What is the value of the investment at the end of 5 years?

  \href{https://youtu.be/0_4bwUvlFeg}{Click here for the solution}
\item
  A loan of \$10,000 is repaid over 4 years. For the first 2 years, the interest rate is 5\% annually, and then it changes to 7\% annually. What is the total amount owed at the end of 4 years?

  \href{https://youtu.be/DjsulaiwpTc}{Click here for the solution}
\item
  An investment of \$2,500 grows to \$3,200 under continuous compounding over 5 years. What is the annual interest rate?

  \href{https://youtu.be/evQHobJj7GY}{Click here for the solution}
\item
  What is the future value of \$1,200 invested for 6 years at an interest rate of 4.2\% compounded continuously?

  \href{https://youtu.be/KuNy6RN3H3k}{Click here for the solution}
\end{enumerate}

\subsection*{Additional Problems:}\label{additional-problems-11}
\addcontentsline{toc}{subsection}{Additional Problems:}

Additional problems that are typically done in class (with video solutions) can be found here:

\begin{itemize}
\item
  \href{https://theelementsmath.github.io/M114/compound-interest.html\#equivalent-payments-and-payment-streams}{Equivalent Payments and Payment Streams}
\item
  \href{https://theelementsmath.github.io/M114/compound-interest.html\#changing-interest-rates-1}{Changing Interest Rates}
\item
  \href{https://theelementsmath.github.io/M114/compound-interest.html\#continuously-compounded-interest}{Continuously Compounded Interest}
\end{itemize}

\part{Annuities}\label{part-annuities}

\chapter{Annuity Basics}\label{annuity-basics}

\section*{Lecture 13: Annuity Basics}\label{lecture-13-annuity-basics}
\addcontentsline{toc}{section}{Lecture 13: Annuity Basics}

\subsection*{Learning Outcomes:}\label{learning-outcomes-12}
\addcontentsline{toc}{subsection}{Learning Outcomes:}

\begin{enumerate}
\def\labelenumi{\arabic{enumi}.}
\tightlist
\item
  Define an \textbf{ordinary simple annuity} and identify the key components of an ordinary simple annuity:

  \begin{itemize}
  \tightlist
  \item
    Payment amount (PMT)
  \item
    Interest rate per period (\(i\))
  \item
    Number of payment periods (\(n\))
  \item
    Present value (PV)
  \item
    Future value (FV)
  \end{itemize}
\item
  Calculate the \textbf{future value} of an ordinary simple annuity using the formula:
  \[
   FV = PMT \left( \frac{(1 + i)^n - 1}{i} \right)
   \]
\item
  Calculate the \textbf{present value} of an ordinary simple annuity using the formula:
  \[
   PV = PMT \left( \frac{1 - (1 + i)^{-n}}{i} \right)
   \]
\item
  Interpret the financial meaning of present and future values in real-world contexts, such as loans, savings plans, and retirement funds.
\item
  Use financial technology tools (e.g., calculators or spreadsheets) to compute annuity values accurately and efficiently.
\end{enumerate}

\subsection*{Review Problems From Last Lecture:}\label{review-problems-from-last-lecture-11}
\addcontentsline{toc}{subsection}{Review Problems From Last Lecture:}

\begin{enumerate}
\def\labelenumi{\arabic{enumi}.}
\tightlist
\item
  What single payment today is equivalent to two future payments of \$600 in 1 year and \$700 in 2 years, assuming 6\% annual compound interest?

  \href{https://youtu.be/heaWuNOF9Wk}{Click here for the solution}
\item
  An investment of \$5,000 earns 4\% interest compounded annually for the first 2 years and then 6\% compounded annually for the next 3 years. What is the value of the investment at the end of 5 years?

  \href{https://youtu.be/urUERAWX47I}{Click here for the solution}
\end{enumerate}

\subsection*{Lecture Notes:}\label{lecture-notes-12}
\addcontentsline{toc}{subsection}{Lecture Notes:}

Lecture material for this class come from Sections 6.1 -- 6.2 and can be found below. This material is considered review material and so it is not covered in depth.

\begin{enumerate}
\def\labelenumi{\arabic{enumi}.}
\tightlist
\item
  \textbf{Video: \href{https://youtu.be/QSn8-A0wNZo}{Ordinary Simple Annuities}}\\
  An \textbf{ordinary simple annuity} is a series of equal payments made at regular intervals, where:

  \begin{itemize}
  \tightlist
  \item
    Payments are made at the end of each period.
  \item
    Interest is calculated using simple interest or consistent compound interest rates.
  \end{itemize}
\item
  \textbf{Video: \href{https://youtu.be/UGp2LF-J0Mo}{Present Value and Future Value}}\\
  The present value (PV) and future value (FV) of an ordinary annuity are given by:
  \[
   PV = PMT \left( \frac{1 - (1 + i)^{-n}}{i} \right)
   \]
  \[
   FV = PMT \left( \frac{(1 + i)^n - 1}{i} \right)
   \] where:

  \begin{itemize}
  \tightlist
  \item
    \(PMT\): periodic payment
  \item
    \(i = \frac{r}{m}\): interest rate per period
  \item
    \(n = m \cdot t\): total number of payments
  \item
    Used to find the accumulated value of all payments at the beginning or end of the term.
  \end{itemize}
\item
  Steps for Solving Annuity Problems:

  \begin{itemize}
  \tightlist
  \item
    Identify whether you're solving for FV or PV.
  \item
    Determine \(PMT\), \(r\), \(m\), \(t\), and calculate \(i\) and \(n\).
  \item
    Substitute known values into the appropriate formula.
  \item
    Solve using a calculator or algebraically.
    Assumptions:
  \item
    Interest rate remains constant over the term.
  \item
    Payments are equal and made at regular intervals.
  \item
    Compounding matches the payment frequency.
  \end{itemize}
\end{enumerate}

\subsection*{Lecture Problems:}\label{lecture-problems-12}
\addcontentsline{toc}{subsection}{Lecture Problems:}

\begin{enumerate}
\def\labelenumi{\arabic{enumi}.}
\tightlist
\item
  You deposit \$200 at the end of each month into an account that pays 6\% annual interest, compounded monthly. What is the future value after 5 years?

  \href{https://youtu.be/OAXT2ToWdBA}{Click here for the solution}
\item
  A company contributes \$1,000 at the end of every quarter into a sinking fund earning 4\% compounded quarterly. What will be the total amount in the fund after 10 years?

  \href{https://youtu.be/VBQIpujkc1M}{Click here for the solution}
\item
  You wish to borrow money and agree to repay it with month end payments of \$500 over 3 years. If the interest rate is 6\% compounded monthly, what is the present value of the loan?

  \href{https://youtu.be/0wgCXQdIBXo}{Click here for the solution}
\item
  A business wants to replace an investment that pays \$2,000 at the end of every 6 months for 8 years. If money is worth 7\% compounded semi-annually, what is the current value of the investment?

  \href{https://youtu.be/cceWAK1oBOQ}{Click here for the solution}
\end{enumerate}

\subsection*{Additional Problems:}\label{additional-problems-12}
\addcontentsline{toc}{subsection}{Additional Problems:}

Additional problems that are typically done in class (with video solutions) can be found here:

\begin{itemize}
\tightlist
\item
  \href{https://theelementsmath.github.io/M114/annuity-basics.html\#annuities}{Annuities}
\end{itemize}

\section*{Lecture 14: Annuities II}\label{lecture-14-annuities-ii}
\addcontentsline{toc}{section}{Lecture 14: Annuities II}

\subsection*{Learning Outcomes:}\label{learning-outcomes-13}
\addcontentsline{toc}{subsection}{Learning Outcomes:}

\begin{enumerate}
\def\labelenumi{\arabic{enumi}.}
\tightlist
\item
  Identify the components of an annuity: present value, future value, interest rate, payment amount (PMT), number of payments (\(n\)), and payment frequency.
\item
  Solve for the periodic payment amount (PMT) in an ordinary general annuity given all other variables.
\item
  Determine the number of payments (\(n\)) in an ordinary general annuity when the other variables are known.
\item
  Use financial calculators or software (e.g., Excel) to compute PMT and \(n\) in practical scenarios.
\end{enumerate}

\subsection*{Review Problems From Last Lecture:}\label{review-problems-from-last-lecture-12}
\addcontentsline{toc}{subsection}{Review Problems From Last Lecture:}

\begin{enumerate}
\def\labelenumi{\arabic{enumi}.}
\tightlist
\item
  You deposit \$250 at the end of every month into a savings account that earns 6\% annual interest, compounded monthly. What will be the future value of the annuity after 5 years?

  \href{https://youtu.be/jIm-SqezwN8}{Click here for the solution}
\item
  Find the future value of an ordinary annuity with quarterly payments of \$2500, an interest rate of 5\% compounded quarterly, and a term of 8 years.

  \href{https://youtu.be/agz1d2ru14s}{Click here for the solution}
\item
  What is the present value of an ordinary annuity that pays \$350 at the end of each month for 4 years, if the annual interest rate is 6\%, compounded monthly?

  \href{https://youtu.be/xs2m_aONP-4}{Click here for the solution}
\item
  Calculate the present value of an annuity with month end payments of \$450 for 3 years at an interest rate of 4\% compounded monthly.

  \href{https://youtu.be/0ReGN005_WE}{Click here for the solution}
\end{enumerate}

\subsection*{Lecture Notes:}\label{lecture-notes-13}
\addcontentsline{toc}{subsection}{Lecture Notes:}

Lecture material for this class come from Section 6.3 and can be found below. This material is considered review material and so it is not covered in depth.

\begin{enumerate}
\def\labelenumi{\arabic{enumi}.}
\setcounter{enumi}{4}
\tightlist
\item
  \textbf{Video: \href{https://youtu.be/WV62y3hfJm03}{Calculating \(PMT\)}}\\
  When the PV or FV of an annuity is known, you can calculate the PMT using the formulas below.
  When \textbf{Future Value} is known:
  \[
   PMT = \frac{FV \cdot i}{(1 + i)^n - 1}
   \]
  When \textbf{Present Value} is known:
  \[
   PMT = \frac{PV \cdot i}{1 - (1 + i)^{-n}}
   \]
\item
  \textbf{Video: \href{https://youtu.be/K-wqThNzu2A}{Calculating \(n\)}}\\
  When the PV or FV of an annuity is known, you can calculate the \(n\) using the formulas below.
  When \textbf{Future Value} is known:
  \[
   n = \frac{\ln\left(\frac{FV \cdot i}{PMT} + 1\right)}{\ln(1 + i)}
   \]
  When \textbf{Present Value} is known:
  \[
   n = \frac{-\ln\left(1-\frac{i \cdot PV}{PMT}\right)}{\ln(1 + i)}
   \]

  \begin{itemize}
  \tightlist
  \item
    \(i\) is the periodic interest rate: \(i = \frac{r}{m}\), where \(r\) is the annual nominal interest rate, and \(m\) is the number of compounding periods per year.
  \item
    \(n\) is the total number of payments: \(n = \text{years} \times m\)

    \item    *

    \textbackslash end\{itemize\}
  \end{itemize}
\item
  \textbf{Video: \href{https://youtu.be/eIW1nf-UtkI}{Solving for the rate}}\\
  To calculate the annual interest rate or the periodic interest rate for an annuity, make sure you use a financial calculator.
\end{enumerate}

\subsection*{Lecture Problems:}\label{lecture-problems-13}
\addcontentsline{toc}{subsection}{Lecture Problems:}

\begin{enumerate}
\def\labelenumi{\arabic{enumi}.}
\tightlist
\item
  You want to accumulate \$10,000 in 3 years by making monthly deposits into an account that earns 6\% annual interest, compounded monthly. How much should you deposit at the end of each month?

  \href{https://youtu.be/1koRK_dWxOM}{Click here for the solution}
\item
  You borrow \$12,000 to be repaid monthly over 5 years at 6\% annual interest compounded monthly. What is the monthly payment?

  \href{https://youtu.be/iTRNLXzGGvA}{Click here for the solution}
\item
  You contribute \$250 at the end of each month into an account earning 6\% annual interest compounded monthly. How many months will it take to accumulate \$8,000?

  \href{https://youtu.be/g7Z2sU2stCY}{Click here for the solution}
\item
  You take out a loan of \$5,000 with monthly payments of \$150 at an annual interest rate of 6\% compounded monthly. How many months will it take to repay the loan?

  \href{https://youtu.be/CAB2PJszt5I}{Click here for the solution}
\end{enumerate}

\subsection*{Additional Problems:}\label{additional-problems-13}
\addcontentsline{toc}{subsection}{Additional Problems:}

Additional problems that are typically done in class (with video solutions) can be found here:

\begin{itemize}
\tightlist
\item
  \href{https://theelementsmath.github.io/M114/annuity-basics.html\#calculating-the-periodic-payment-and-number-of-payments}{Calculating \(PMT\) and \(n\)}
\end{itemize}

\section*{Lecture 15: Ordinary General Annuities}\label{lecture-15-ordinary-general-annuities}
\addcontentsline{toc}{section}{Lecture 15: Ordinary General Annuities}

\subsection*{Learning Outcomes:}\label{learning-outcomes-14}
\addcontentsline{toc}{subsection}{Learning Outcomes:}

\begin{enumerate}
\def\labelenumi{\arabic{enumi}.}
\tightlist
\item
  Define and distinguish between nominal, effective, and equivalent interest rates.
\item
  Convert a nominal annual interest rate compounded \(m_1\) times per year to its equivalent effective annual rate using the formula:
  \[
   i_{\text{eff}} = \left(1 +i_i\right)^{m_1} - 1
   \]
  and an equivalent form compounded \(m_2\) times per year using the formula:
  \[
   i_2 =  \left( (1 + i_{1})^{m_1/m_2} \right) - 1 .
   \]
\item
  Identify the characteristics of an ordinary general annuity, including differing payment and compounding frequencies.
\item
  Calculate the present value (PV), future value (FV), periodic payment (PMT) or number of payments (\(n\)) of an ordinary general annuity.
\item
  Use a financial calculator to solve general annuity problems.
\end{enumerate}

\subsection*{Review Problems From Last Lecture:}\label{review-problems-from-last-lecture-13}
\addcontentsline{toc}{subsection}{Review Problems From Last Lecture:}

\begin{enumerate}
\def\labelenumi{\arabic{enumi}.}
\tightlist
\item
  You want to save \$10,000 in 5 years in an account that pays 6\% interest compounded annually. How much should you deposit at the end of each year?

  \href{https://youtu.be/eyVyq_2Wgu4}{Click here for the solution}
\item
  A loan of \$25,000 is to be repaid in monthly payments over 4 years at an annual interest rate of 9\%, compounded monthly. What is the monthly payment?

  \href{https://youtu.be/wbn7X2z9JGM}{Click here for the solution}
\item
  You deposit \$200 at the end of each month into a savings account earning 5\% interest compounded monthly. How long will it take to accumulate \$10,000?

  \href{https://youtu.be/Vbx8oxYbk-A}{Click here for the solution}
\item
  You take a loan of \$15,000 and agree to repay it with annual payments of \$3,500 at 7\% interest compounded annually. How many years will it take to repay the loan?

  \href{https://youtu.be/XDfm1eRTefE}{Click here for the solution}
\end{enumerate}

\subsection*{Lecture Notes:}\label{lecture-notes-14}
\addcontentsline{toc}{subsection}{Lecture Notes:}

Lecture material for this class come from Sections 6.1, 6.4, and 6.5 and can be found below. This material is considered review material and so it is not covered in depth.

\begin{enumerate}
\def\labelenumi{\arabic{enumi}.}
\tightlist
\item
  \textbf{Video: \href{https://youtu.be/-m7XgzLWk2g}{Equivalent Rates}}\\
  An \textbf{equivalent interest rate} is a rate that gives the same future value as another rate but is compounded at a different frequency. An equivalent periodic interest rate is given by \(i_2 = \left(1 + i_1\right)^{m_1/m_2} - 1\).
\item
  \textbf{Video: \href{https://youtu.be/8KY9aTEcZog}{Ordinary General Annuities}}\\
  An \textbf{ordinary general annuity} is a financial arrangement where equal payments (PMTs) are made at the end of each period, but the payment period and the interest compounding period do not match.

  \begin{itemize}
  \tightlist
  \item
    Payments are made at the end of each payment interval.
  \item
    The payment interval differs from the interest conversion period.
  \item
    The interest rate must be converted to match the payment frequency.\\
    Key definitions:

    \begin{itemize}
    \tightlist
    \item
      \(i_1\): interest rate per compounding period
    \item
      \(m_1\): number of compounding periods per year
    \item
      \(m_2\): number of payment periods per year
    \item
      \(i_1 = \frac{rate}{m_1}\): interest rate per compounding period
    \item
      \(i_2 = \left(1 + i_1\right)^{m_1/m_2} - 1\): \textbf{equivalent rate per payment period}
    \item
      \(n\): total number of payments
    \item
      \(\text{PMT}\): periodic payment
    \end{itemize}
  \end{itemize}
\item
  Steps to Solve General Annuity Problems

  \begin{itemize}
  \tightlist
  \item
    Identify the nominal annual rate \(i_1\), compounding frequency \(m_1\), and payment frequency \(m_2\).
  \item
    Compute the interest per compounding period: \(i_1 = \frac{rate}{m_1}\).
  \item
    Convert \(i_2\) to an equivalent payment period rate \(i_2 = \left(1 + i_1 \right)^{m_1/m_2} - 1\).
  \item
    Determine total number of payments \(n = \text{years} \times m_2\).
  \item
    Apply appropriate formula (FV or PV).
  \end{itemize}
\end{enumerate}

\subsection*{Lecture Problems:}\label{lecture-problems-14}
\addcontentsline{toc}{subsection}{Lecture Problems:}

\begin{enumerate}
\def\labelenumi{\arabic{enumi}.}
\tightlist
\item
  You deposit \$400 at the end of every month into an investment account that earns 6\% interest compounded quarterly. What will be the future value of the annuity after 7 years?

  \href{https://youtu.be/59s8zc-RoA4}{Click here for the solution}
\item
  You want to buy a car and can afford to pay \$350 at the end of every month for 5 years. If the dealer offers financing at 8\% compounded quarterly, what is the present value of the loan?

  \href{https://youtu.be/wi6EyYtNxRA}{Click here for the solution}
\item
  You want to accumulate \$20,000 in 3 years for a vacation by making quarterly payments into an account that earns 5\% interest compounded monthly. What should each payment be?

  \href{https://youtu.be/xtDl74YNEW4}{Click here for the solution}
\item
  You invest \$300 every 2 months into a fund that will grow to \$10,000 in 5 years. If interest is compounded monthly, what nominal annual interest rate is being earned?

  \href{https://youtu.be/rYzyGRqF71c}{Click here for the solution}
\item
  You invest \$500 at the end of each quarter into an account earning 6\% compounded monthly. How many quarters will it take for the account to reach \$25,000?

  \href{https://youtu.be/Khm8FadwIGA}{Click here for the solution}
\end{enumerate}

\subsection*{Additional Problems:}\label{additional-problems-14}
\addcontentsline{toc}{subsection}{Additional Problems:}

Additional problems that are typically done in class (with video solutions) can be found here:

\begin{itemize}
\item
  \href{https://theelementsmath.github.io/M114/annuity-basics.html\#effective-and-equivalent-interest-rates}{Effective and Equivalent Interest Rates}
\item
  \href{https://theelementsmath.github.io/M114/annuity-basics.html\#ordinary-general-annuities}{Ordinary General Annuities}
\item
  \href{https://theelementsmath.github.io/M114/annuity-basics.html\#using-the-ti-baii-plus-calculator-for-annuities}{TI BAII plus Calculator for Annuities}
\end{itemize}

\part{Investments}\label{part-investments}

\chapter{Mortgages}\label{mortgages}

\section*{Lecture 18: Mortgages}\label{lecture-18-mortgages}
\addcontentsline{toc}{section}{Lecture 18: Mortgages}

\subsection*{Learning Outcomes:}\label{learning-outcomes-15}
\addcontentsline{toc}{subsection}{Learning Outcomes:}

\begin{enumerate}
\def\labelenumi{\arabic{enumi}.}
\tightlist
\item
  Define amortization and explain its purpose in loan repayment.
\item
  Identify the components of an amortization schedule: payment number, payment amount, interest portion, principal portion, and remaining balance.
\item
  Calculate the periodic payment (PMT) for a fully amortized loan using the appropriate time value of money formulas.
\item
  Construct a complete amortization table for a fixed-rate loan using given loan terms.
\item
  Distinguish between interest and principal components in each payment.
\end{enumerate}

\subsection*{Review Problems From Last Lecture:}\label{review-problems-from-last-lecture-14}
\addcontentsline{toc}{subsection}{Review Problems From Last Lecture:}

\begin{enumerate}
\def\labelenumi{\arabic{enumi}.}
\tightlist
\item
  A retiree plans to purchase a deferred annuity that will pay \$2,000 per month, with the first payment exactly 10 years from now, and will continue for 20 years. The annuity earns an interest rate of 6\% compounded monthly. What is the present value of the annuity (i.e., how much should the retiree invest today to receive this income stream)?

  \href{https://youtu.be/HVz-VL22-JI}{Click here for the solution}
\item
  An investor wants to purchase a deferred perpetuity that will pay \$5,000 annually, starting 8 years from today and continuing forever. The annual interest rate is 7\%, compounded annually. What is the present value of this investment today?

  \href{https://youtu.be/FHhqwkhFg9A}{Click here for the solution}
\item
  Maria decides to invest in her RRSP to prepare for retirement. She contributes \$6,000 at the end of each year for 20 years. The account earns an annual interest rate of 5\%, compounded annually. After the 20 years of contributions, Maria stops adding money to the RRSP and allows the investment to grow with compound interest for another 10 years without any withdrawals. At the end of this 10-year accumulation period, Maria begins withdrawing \$25,000 at the end of each year for her retirement. The account continues to earn 5\% interest, compounded annually, during the withdrawal phase.

  \begin{itemize}
  \tightlist
  \item
    What will be the value of Maria's RRSP at the time she stops contributing (i.e., after 20 years)?
  \item
    What will be the value of the RRSP at the beginning of the withdrawal phase (i.e., after the additional 10 years of interest accumulation)?
  \item
    For how many years will Maria be able to withdraw \$25,000 annually before the RRSP is depleted?

    \href{https://youtu.be/fwglKbNsHZ4}{Click here for the solution}
  \end{itemize}
\end{enumerate}

\subsection*{Lecture Notes:}\label{lecture-notes-15}
\addcontentsline{toc}{subsection}{Lecture Notes:}

Lecture material for this class come from Sections 8.1 and can be found below. This material is considered review material and so it is not covered in depth.

\begin{enumerate}
\def\labelenumi{\arabic{enumi}.}
\setcounter{enumi}{4}
\tightlist
\item
  \textbf{Video: \href{https://youtu.be/vdxdyhyXa8Q}{Systems of Equations I}}\\
  \textbf{Video: \href{https://youtu.be/tZpRI_6_Og0}{Systems of Equations II}}\\
  An \textbf{amortization table} shows how a loan is repaid over time through regular periodic payments. Each payment consists of both interest and principal components.\\
  Purpose

  \begin{itemize}
  \tightlist
  \item
    To track how much of each payment goes toward interest vs.~principal.
  \item
    To monitor the remaining loan balance after each payment.
  \item
    To understand the cost of borrowing over time.
  \end{itemize}
\end{enumerate}

Key Components of an Amortization Table

\begin{itemize}
\tightlist
\item
  Payment Number (e.g., 1, 2, 3, \ldots, \(n\))
\item
  Payment Amount (fixed for a level-payment loan)
\item
  Interest Portion = Current balance \(\times\) periodic interest rate
\item
  Principal Portion = Total payment \(-\) interest
\item
  Remaining Balance = Previous balance \(-\) principal paid
\end{itemize}

Important Notes

\begin{itemize}
\tightlist
\item
  The interest portion decreases over time.
\item
  The principal portion increases over time.
\item
  The loan is fully paid off at the end of the amortization period.
\item
  Total interest paid can be significant, especially for long-term loans.
\end{itemize}

\subsection*{Lecture Problems:}\label{lecture-problems-15}
\addcontentsline{toc}{subsection}{Lecture Problems:}

\begin{enumerate}
\def\labelenumi{\arabic{enumi}.}
\tightlist
\item
  You borrow \$10,000 and will repay it with \$3000 payments at the end of each year. Create a loan repayment schedule. Use 5\% annually compounded interest.

  \href{https://youtu.be/k9lk4D2iPtc}{Click here for the solution}
\item
  ou need \$450,000 for your mortgage. You repay this back over some period of time with biweekly payments of \$1,400. Create a loan repayment schedule detailing the first 3 lines and the last 2 lines of the repayment schedule. The mortgage rate is 4\% compounded semi-annually.

  \href{https://youtu.be/LuDq5KVlKeg}{Click here for the solution}
\end{enumerate}

\subsection*{Additional Problems:}\label{additional-problems-15}
\addcontentsline{toc}{subsection}{Additional Problems:}

Additional problems that are typically done in class (with video solutions) can be found here:

\begin{itemize}
\tightlist
\item
  \href{https://theelementsmath.github.io/M114/mortgages.html\#mortgage-fundamentals}{Mortgages}
\end{itemize}

\chapter{More on Annuities}\label{more-on-annuities}

\section*{Lecture 16: Annuities Due}\label{lecture-16-annuities-due}
\addcontentsline{toc}{section}{Lecture 16: Annuities Due}

\subsection*{Learning Outcomes:}\label{learning-outcomes-16}
\addcontentsline{toc}{subsection}{Learning Outcomes:}

\begin{enumerate}
\def\labelenumi{\arabic{enumi}.}
\tightlist
\item
  Define an \textbf{annuity due} and distinguish it from an \textbf{ordinary annuity}.
\item
  Calculate the \textbf{future value} of an annuity due using the appropriate formula.
\item
  Calculate the \textbf{present value} of an annuity due.
\item
  Solve for unknowns in annuity due problems, including:

  \begin{itemize}
  \tightlist
  \item
    Periodic payment (\(\text{PMT}\))
  \item
    Number of periods (\(n\))
  \item
    Interest rate (\(i\))
  \end{itemize}
\item
  Interpret and construct \textbf{timelines} to visualize cash flows of annuities due.
\item
  Apply time value of money principles in analyzing annuities due in financial planning and decision-making.
\end{enumerate}

\subsection*{Review Problems From Last Lecture:}\label{review-problems-from-last-lecture-15}
\addcontentsline{toc}{subsection}{Review Problems From Last Lecture:}

\begin{enumerate}
\def\labelenumi{\arabic{enumi}.}
\tightlist
\item
  You are planning to take out a loan and will make monthly payments of \$600 for 6 years. The interest rate is 7\% compounded quarterly. What is the present value of the loan?

  \href{https://youtu.be/LudOJKtJSTU}{Click here for the solution}
\item
  You invest \$250 every 2 months into an account that earns 6.5\% interest compounded monthly. What will be the value of the investment after 8 years?

  \href{https://youtu.be/rjgtfe-PjDs}{Click here for the solution}
\item
  You want to accumulate \$15,000 over 5 years by making monthly payments into an account earning 4.8\% compounded semi-annually. What should each monthly payment be?

  \href{https://youtu.be/__aieU0ELCs}{Click here for the solution}
\item
  You invest \$150 every quarter and end up with \$5,000 after 7 years. If interest is compounded monthly, what is the nominal annual interest rate?

  \href{https://youtu.be/as5FICA8nLQ}{Click here for the solution}
\item
  A financial institution offers a nominal annual interest rate of 6\% compounded quarterly.

  \begin{itemize}
  \tightlist
  \item
    What is the \textbf{effective annual rate (EAR)}?
  \item
    What is the \textbf{equivalent nominal annual rate} if interest were instead compounded monthly?
  \item
    Compare both options: 6\% compounded quarterly vs.~the equivalent nominal rate you found in part (b) compounded monthly. Which compounding method yields more interest over a year?

    \href{https://youtu.be/cNuDhyZxDnQ}{Click here for the solution}
  \end{itemize}
\end{enumerate}

\subsection*{Lecture Notes:}\label{lecture-notes-16}
\addcontentsline{toc}{subsection}{Lecture Notes:}

Lecture material for this class come from Sections 7.1 and can be found below. This material is considered review material and so it is not covered in depth.

\begin{enumerate}
\def\labelenumi{\arabic{enumi}.}
\tightlist
\item
  \textbf{Video: \href{https://youtu.be/xalP41MWGSg}{Annuities Due}}\\
  An \textbf{annuity due} is a series of equal payments made at the beginning of each period, rather than at the end (as in an ordinary annuity). This type of annuity is common in situations like rent, lease payments, and insurance premiums.
\item
  \textbf{Video: \href{https://youtu.be/knuOJ03sshA}{PV and FV of Annuities Due}}\\
  Future Value of an Annuity Due
  \[
  FV_{\text{due}} = \text{PMT} \cdot \left[\frac{(1 + i)^n - 1}{i} \right] \cdot (1 + i)
  \]
  This is equivalent to:
  \[
  FV_{\text{due}} = FV_{\text{ordinary}} \cdot (1 + i)
  \]
\item
  Present Value of an Annuity Due
  \[
  PV_{\text{due}} = \text{PMT} \cdot \left[\frac{1 - (1 + i)^{-n}}{i} \right] \cdot (1 + i)
  \]
  Again:
  \[
  PV_{\text{due}} = PV_{\text{ordinary}} \cdot (1 + i)
  \]
\item
  Solving Annuity Due Problems

  \begin{itemize}
  \tightlist
  \item
    Identify whether payments are made at the beginning (annuity due) or end (ordinary annuity) of each period.
  \item
    Determine:

    \begin{itemize}
    \tightlist
    \item
      \(i\): periodic interest rate
    \item
      \(n\): number of payment periods
    \item
      \(\text{PMT}\): periodic payment
    \end{itemize}
  \item
    Use the correct formula for present or future value.
  \item
    Multiply by \((1 + i)\) if using ordinary annuity formulas to adjust for annuity due.
  \end{itemize}
\end{enumerate}

\subsection*{Lecture Problems:}\label{lecture-problems-16}
\addcontentsline{toc}{subsection}{Lecture Problems:}

\begin{enumerate}
\def\labelenumi{\arabic{enumi}.}
\tightlist
\item
  You deposit \$150 at the beginning of every month into a savings account that earns 5\% interest compounded monthly. What will be the future value of the annuity after 10 years?

  \href{https://youtu.be/2ZHW43unmi8}{Click here for the solution}
\item
  You plan to prepay a 4-year lease with annual payments of \$3,000, made at the beginning of each year. If the interest rate is 6\% compounded annually, what is the present value of the lease?

  \href{https://youtu.be/oiKuXhGN_RY}{Click here for the solution}
\item
  An insurance policy requires you to pay \$1,200 at the beginning of each quarter for 5 years. If the insurer uses an interest rate of 4\% compounded quarterly, what is the present value of your policy payments?

  \href{https://youtu.be/JplE47c91xU}{Click here for the solution}
\item
  You are saving for a car by depositing \$500 at the beginning of each quarter into an account earning 6\% interest compounded quarterly. What will be the value of your account after 6 years?

  \href{https://youtu.be/RydqSKjDjBg}{Click here for the solution}
\end{enumerate}

\subsection*{Additional Problems:}\label{additional-problems-16}
\addcontentsline{toc}{subsection}{Additional Problems:}

Additional problems that are typically done in class (with video solutions) can be found here:

\begin{itemize}
\tightlist
\item
  \href{https://theelementsmath.github.io/M114/more-on-annuities.html\#annuities-due}{Annuities Due}
\end{itemize}

\section*{Lecture 17: Special Case Annuities}\label{lecture-17-special-case-annuities}
\addcontentsline{toc}{section}{Lecture 17: Special Case Annuities}

\subsection*{Learning Outcomes:}\label{learning-outcomes-17}
\addcontentsline{toc}{subsection}{Learning Outcomes:}

\begin{enumerate}
\def\labelenumi{\arabic{enumi}.}
\tightlist
\item
  Define a deferred annuity and construct timelines to represent deferred annuity cash flows.
\item
  Solve problems involving both ordinary and annuity due types of deferred annuities.
\item
  Define a perpetuity and identify real-world examples (e.g., preferred stocks, endowments).
\item
  Solve problems involving regular perpetuities and deferred perpetuities.
\item
  Define a constant growth annuity and explain its relevance in financial valuation (e.g., growing retirement withdrawals or dividend models).
\end{enumerate}

\subsection*{Review Problems From Last Lecture:}\label{review-problems-from-last-lecture-16}
\addcontentsline{toc}{subsection}{Review Problems From Last Lecture:}

\begin{enumerate}
\def\labelenumi{\arabic{enumi}.}
\tightlist
\item
  You contribute \$250 at the beginning of each month into a retirement savings account that earns 7.2\% interest compounded monthly. How much will be in the account after 15 years?

  \href{https://youtu.be/e7x69WQbUy8}{Click here for the solution}
\item
  You agree to pay rent of \$1,500 at the beginning of each month for 3 years. If the landlord uses a discount rate of 5\% compounded monthly, what is the present value of the lease agreement?

  \href{https://youtu.be/VqldJMs4-Oo}{Click here for the solution}
\item
  You are saving for a down payment by depositing \$400 at the beginning of every 2 months into an account that earns 5.4\% compounded monthly. What will the account be worth in 6 years?

  \href{https://youtu.be/DHjYuRYo8JI}{Click here for the solution}
\item
  A company offers an employee bonus plan that pays \$5,000 at the beginning of each year for 8 years. If the discount rate is 6.5\% compounded annually, what is the present value of the bonus payments?

  \href{https://youtu.be/pipYbAzRvG4}{Click here for the solution}
\end{enumerate}

\subsection*{Lecture Notes:}\label{lecture-notes-17}
\addcontentsline{toc}{subsection}{Lecture Notes:}

Lecture material for this class come from Sections 7.2, 7.3, 7.4 and 7.5 and can be found below. This material is considered review material and so it is not covered in depth.

\begin{enumerate}
\def\labelenumi{\arabic{enumi}.}
\tightlist
\item
  \textbf{Video: \href{https://youtu.be/S5UgF1tzz6s}{Deferred Annuities}}\\
  A \textbf{deferred annuity} is an annuity where payments begin after a delay or deferment period.\\
  Key Points:

  \begin{itemize}
  \tightlist
  \item
    Payments start after a set number of periods.
  \item
    Present value is calculated in two steps:

    \begin{itemize}
    \tightlist
    \item
      Find the present value at the time just before the first payment.
    \item
      Discount that value back to today.
    \end{itemize}
  \item
    Applies to both ordinary annuities and annuities due.
  \end{itemize}
\item
  \textbf{Video: \href{https://youtu.be/CNq74zq-Jns}{Perpetuities}}\\
  A \textbf{perpetuity} is a series of equal payments made at regular intervals that continue indefinitely.\\
  Formula:
  \[
  PV = \frac{\text{PMT}}{i}, \quad \text{where } i > 0
  \]\\
  Key Points:

  \begin{itemize}
  \tightlist
  \item
    No maturity --- payments go on forever.
  \item
    Present value is finite as long as \(i > 0\).
  \end{itemize}
\item
  \textbf{Video: \href{https://youtu.be/ooUss4RfLHI}{Constant Growth Annuities}}\\
  A \textbf{constant growth annuity} is an annuity where the payments grow at a constant rate \(g\) per period.\\
  Formula:
  \[
  PV = \text{PMT}_1 \cdot \frac{1 - (1+g)^n(1+i)^{-n}}{i-g}
  \]
  \[
  FV = \text{PMT}_1 \cdot \frac{(1+i)^n- (1+g)^n}{i-g}
  \]
  where:

  \begin{itemize}
  \tightlist
  \item
    \(\text{PMT}_1\) is the first payment.
  \item
    \(i\) is the interest rate per period.
  \item
    \(g\) is the growth rate of the payments.
  \item
    \(n\): the number of periods.\\
    Key Points:
  \item
    Payments increase over time: \(\text{PMT}_1, \text{PMT}_1(1+g), \text{PMT}_1(1+g)^2, \dots\)
  \item
    This formula assumes that payments are made at the \textbf{end of each period}, and that \(i > g\).
  \end{itemize}
\end{enumerate}

\subsection*{Lecture Problems:}\label{lecture-problems-17}
\addcontentsline{toc}{subsection}{Lecture Problems:}

\begin{enumerate}
\def\labelenumi{\arabic{enumi}.}
\tightlist
\item
  A preferred stock pays a fixed dividend of \$4.50 per share every year, indefinitely. If investors require a 5.5\% annual return, what is the fair price of the stock?

  \href{https://youtu.be/DcH6X7vmuJI}{Click here for the solution}
\item
  You plan to purchase an investment that will pay \$1,200 at the end of each year for 10 years. However, the first payment will not be received until 5 years from now. If the interest rate is 6\% compounded annually, what is the present value of this deferred annuity today?

  \href{https://youtu.be/KKtTShNruXk}{Click here for the solution}
\item
  You plan to withdraw an income that starts at \$5,000 at the end of the first year and grows at 3\% annually for 20 years. If the interest rate is 7\% compounded annually, what is the present value of these withdrawals at the start of year 1?

  \href{https://youtu.be/MgdL-JOsw38}{Click here for the solution}
\end{enumerate}

\subsection*{Additional Problems:}\label{additional-problems-17}
\addcontentsline{toc}{subsection}{Additional Problems:}

Additional problems that are typically done in class (with video solutions) can be found here:

\begin{itemize}
\item
  \href{https://theelementsmath.github.io/M114/more-on-annuities.html\#deferred-annuities}{Deferred Annuities}
\item
  \href{https://theelementsmath.github.io/M114/more-on-annuities.html\#perpetuities}{Perpetuities}
\item
  \href{https://theelementsmath.github.io/M114/more-on-annuities.html\#constant-growth-annuities}{Constant Growth Annuities}
\item
  \href{https://theelementsmath.github.io/M114/more-on-annuities.html\#special-case-annuities}{Special Case Annuities}
\end{itemize}

  \bibliography{book.bib,packages.bib}

\end{document}
